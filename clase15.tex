\documentclass[11pt]{article}

% --------- Idioma y tipografía ----------
\usepackage[spanish, es-nodecimaldot]{babel}
\usepackage[T1]{fontenc}
\usepackage[utf8]{inputenc}
\usepackage{microtype}
\usepackage{newpxtext,newpxmath} % Palatino + matemáticas
\usepackage[a4paper,margin=1in]{geometry}

% --------- Matemática y utilidades -------
\usepackage{amsmath,mathtools}

% --------- Colores, enlaces, refs --------
\usepackage{xcolor}
\definecolor{accent}{RGB}{10,80,160}
\usepackage[
  colorlinks,
  linkcolor=accent,
  urlcolor=accent,
  citecolor=accent
]{hyperref}
\usepackage[nameinlink]{cleveref}

% --------- Encabezados/Pies --------------
\usepackage{fancyhdr}
\pagestyle{fancy}
\fancyhf{}
\lhead{\textit{Problemas de Programación Lineal}}
\rhead{Ricardo Largaespada}
\cfoot{\thepage}
\setlength{\headheight}{14pt}

% --------- Tablas bonitas ----------------
\usepackage{booktabs,tabularx}
\renewcommand{\arraystretch}{1.12}

% --------- Cajas para "Problema" ---------
\usepackage[most]{tcolorbox}
\tcbset{
  boxsep=5pt,
  arc=2mm,
  colback=gray!2!white,
  colframe=accent!60!black,
  boxrule=0.6pt
}
\newtcbtheorem[auto counter]{problema}{Problema}{
  fonttitle=\bfseries,
  coltitle=white
}{prob}

% --------- Entorno de solución (toggle) ----------
\newif\ifshowsolutions
\showsolutionsfalse % pon \showsolutionstrue para mostrar soluciones

\newenvironment{solucion}[1][]%
{\ifshowsolutions\begin{tcolorbox}[title=Solución, breakable,
  colback=accent!6!white, colframe=accent!80!black,
  boxrule=0.6pt, fonttitle=\bfseries, #1]}%
{\end{tcolorbox}\fi}

% --------- Título ------------------------
\title{\textbf{Problemas sobre Dualidad en Programación Lineal}}
\author{Ricardo Largaespada}
\date{06 de octubre de 2025}

\begin{document}
\maketitle
\thispagestyle{fancy}

% =====================================================
\begin{problema}{Precios sombra en desarrollo y servidores (2 variables)}{}
Una empresa de tecnología está desarrollando dos productos de software: un \textbf{sistema operativo} (\(x_1\)) y una \textbf{aplicación móvil} (\(x_2\)). Cuenta con 200 horas de desarrollo y 150~GB de espacio en servidores para pruebas y despliegue. El objetivo es maximizar las ganancias.

\medskip
\textbf{Datos por unidad producida}
\begin{center}
\begin{tabular}{lccc}
\toprule
& \textbf{Sistema Operativo} (\(x_1\)) & \textbf{Aplicación Móvil} (\(x_2\)) & \textbf{Disponibilidad} \\
\midrule
Horas de desarrollo & 50 & 40 & 200 \\
Espacio en servidores (GB) & 30 & 20 & 150 \\
Ganancia (USD) & 200 & 150 & \\
\bottomrule
\end{tabular}
\end{center}

\textbf{Primal (max):}
\[
\max\; 200x_1+150x_2
\quad \text{s.a.}\quad
\begin{aligned}
50x_1+40x_2 &\le 200,\\
30x_1+20x_2 &\le 150,\\
x_1,x_2 &\ge 0.
\end{aligned}
\]

\textbf{Tareas:}
\begin{enumerate}
  \item Formule el \textbf{dual} e identifique las variables duales \(y_1\) (horas) y \(y_2\) (GB).
  \item \textbf{Calcule} los \emph{precios sombra} óptimos de cada recurso (\(y_1^\star, y_2^\star\)).
  \item Explique el \emph{significado económico} de \(y_1^\star\) y \(y_2^\star\) (variación marginal en la ganancia óptima por una unidad adicional del recurso).
  \item Dado que la empresa puede comprar:
    \begin{itemize}
      \item horas de desarrollo a un costo de \(\$c_h\) por hora,
      \item GB de servidor a un costo de \(\$c_s\) por GB,
    \end{itemize}
    determine \emph{cuál recurso conviene adquirir} comparando \(y^\star\) con los costos \(c_h, c_s\) (regla: invierta si \(y^\star\) supera el costo marginal).
\end{enumerate}
\end{problema}

% =====================================================
\begin{problema}{Expansión de infraestructura en un ISP local (dualidad e inversión)}{}
La empresa \textbf{Redes del Sur} provee Internet en una ciudad de Nicaragua y considera invertir en capacidad adicional para satisfacer demanda.

\medskip
\textbf{Datos}
\begin{itemize}
  \item Capacidad actual de ancho de banda: 400 Mbps (costo adicional: \(\$250\) por Mbps).
  \item Antenas de transmisión: 20 unidades (costo adicional: \(\$2000\) por antena).
  \item Plan Básico \(x_1\): requiere 1 Mbps y 0.5 antenas por usuario; ingreso \(\$30\) por usuario.
  \item Plan Premium \(x_2\): requiere 3 Mbps y 1 antena por usuario; ingreso \(\$70\) por usuario.
\end{itemize}

\textbf{Primal (max):}
\[
\max\; 30x_1+70x_2
\quad \text{s.a.}\quad
\begin{aligned}
x_1+3x_2 &\le 400 && \text{(ancho de banda, dual \(y_1\))},\\
0.5x_1+x_2 &\le 20 && \text{(antenas, dual \(y_2\))},\\
x_1,x_2 &\ge 0.
\end{aligned}
\]

\textbf{Tareas:}
\begin{enumerate}
  \item Formule el \textbf{dual} (min) e interprete \(y_1\) y \(y_2\) como \emph{precios sombra} por Mbps y por antena.
  \item \textbf{Determine} a precios sombra óptimos si es \emph{rentable} invertir en:
    \begin{itemize}
      \item ancho de banda adicional a \(\$250\)/Mbps,
      \item antenas adicionales a \(\$2000\)/unidad.
    \end{itemize}
    (Regla: invierta si \(y_i^\star >\) costo marginal del recurso \(i\).)
  \item Analice cómo \(y_1^\star\) y \(y_2^\star\) guían decisiones de \emph{pricing} y \emph{mix} de planes (Básico vs. Premium) bajo restricciones activas/holgadas.
\end{enumerate}
\end{problema}

% =====================================================
\begin{problema}{Optimización de recursos en TechNica (dualidad y expansión)}{}
La empresa \textbf{TechNica} desarrolla software y hardware, y evalúa contratar más personal y adquirir más equipos para aumentar producción.

\medskip
\textbf{Datos}
\begin{itemize}
  \item Personal: 50 empleados (costo adicional de referencia: \(\$800\) por empleado/mes).
  \item Equipos: 30 unidades (costo adicional de referencia: \(\$1500\) por equipo).
  \item Proyecto Software \(x_1\): requiere 2 empleados y 1 equipo por unidad; ingreso \(\$5000\) por unidad.
  \item Proyecto Hardware \(x_2\): requiere 3 empleados y 2 equipos por unidad; ingreso \(\$8000\) por unidad.
\end{itemize}

\textbf{Primal (max):}
\[
\max\; 5000x_1+8000x_2
\quad \text{s.a.}\quad
\begin{aligned}
2x_1+3x_2 &\le 50 && \text{(empleados, dual \(y_1\))},\\
x_1+2x_2 &\le 30 && \text{(equipos, dual \(y_2\))},\\
x_1,x_2 &\ge 0.
\end{aligned}
\]

\textbf{Tareas:}
\begin{enumerate}
  \item Formule el \textbf{dual} y determine \(y_1^\star, y_2^\star\).
  \item \textbf{Decida} si es rentable \emph{contratar más personal} (\$800/empleado) y/o \emph{adquirir más equipos} (\$1500/equipo) comparando con \(y_1^\star\) y \(y_2^\star\).
  \item Explique el \emph{significado económico} de un precio sombra positivo y el papel de la \emph{complementariedad holgura–dual} en la selección del portafolio \(x_1, x_2\).
\end{enumerate}
\end{problema}

% ======== (Opcional) esqueletos de solución para activar con \showsolutionstrue ========
\begin{solucion}
\textbf{Pautas generales de corrección (para los tres problemas):}
\begin{itemize}
  \item Verificar formulación dual correcta y unidades de \(y_i\).
  \item Aplicar condición de \emph{inversión marginal}: invertir si \(y_i^\star\) supera el costo marginal del recurso.
  \item Usar complementariedad: si una restricción está holgada, su \(y_i^\star=0\).
\end{itemize}
\end{solucion}

\fi
\end{document}
