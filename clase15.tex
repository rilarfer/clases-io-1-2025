\documentclass{article}
\usepackage{amsmath}
\usepackage{amssymb}
\usepackage{geometry}
\usepackage{xcolor}
\usepackage{mdframed}
\geometry{a4paper, margin=1in}

\title{Problemas sobre Dualidad en Programación Lineal}
\author{Ricardo Largaespada}
\date{15 de octubre 2024}

\newmdenv[
  backgroundcolor=blue!5,
  linecolor=blue,
  linewidth=1pt,
  roundcorner=5pt,
  skipabove=\baselineskip,
  skipbelow=\baselineskip
]{problem}

\begin{document}

\maketitle

\section*{Problemas}

\begin{problem}
\subsection*{Problema 1:} Una empresa de tecnología está desarrollando dos productos de software: un \textbf{sistema operativo} y una \textbf{aplicación móvil}. Cuenta con 200 horas de desarrollo y 150 GB de espacio en servidores para pruebas y despliegue. El objetivo es maximizar las ganancias dadas las siguientes restricciones:

\begin{itemize}
    \item \textbf{Sistema Operativo:} Requiere 50 horas de desarrollo y 30 GB de espacio por unidad producida.
    \item \textbf{Aplicación Móvil:} Requiere 40 horas de desarrollo y 20 GB de espacio por unidad producida.
    \item \textbf{Ganancias:} Sistema Operativo = \$200 por unidad, Aplicación Móvil = \$150 por unidad.
\end{itemize}

\textbf{Preguntas:}
\begin{enumerate}
    \item Identifique y calcule los precios sombra de cada recurso (horas de desarrollo y espacio en servidores).
    \item Explique el significado económico de los precios sombra en este contexto.
    \item Determine cuál recurso debería adquirir la empresa para maximizar sus beneficios.
\end{enumerate}

\end{problem}

\begin{problem}
\subsection*{Problema 2: Expansión de Infraestructura en un Proveedor de Internet Local}

La empresa \textbf{Redes del Sur} provee servicios de Internet en una ciudad de Nicaragua y está considerando invertir en capacidad adicional para satisfacer la creciente demanda.

\textbf{Datos:}
\begin{itemize}
    \item Capacidad actual de ancho de banda: 400 Mbps (costo adicional de 250 dólares por Mbps).
    \item Número actual de antenas de transmisión: 20 unidades (costo adicional de 2000 dólares por antena).
    \item Plan Básico \( x_1 \) requiere 1 Mbps y 0.5 antenas por usuario.
    \item Plan Premium \( x_2 \) requiere 3 Mbps y 1 antena por usuario.
    \item Ingreso por usuario del Plan Básico: 30 dólares.
    \item Ingreso por usuario del Plan Premium: 70 dólares.
\end{itemize}

\textbf{Formulación del Primal:}
\[
\max \quad 30x_1 + 70x_2
\]
Sujeto a:
\[
\begin{aligned}
    x_1 + 3x_2 &\leq 400 \\
    0.5x_1 + x_2 &\leq 20 \\
    x_1, x_2 &\geq 0
\end{aligned}
\]

\textbf{Pregunta:}
\begin{enumerate}
    \item Determine si es rentable invertir en ancho de banda adicional y antenas para maximizar las ganancias.
    \item Explique cómo las variables duales reflejan el valor económico de aumentar la capacidad de ancho de banda y el número de antenas. ¿Cómo afectan los precios sombra las decisiones de inversión?
\end{enumerate}

\end{problem}

\begin{problem}
\subsection*{Problema 3: Optimización de Recursos en una Empresa Tecnológica Nicaragüense}

La empresa \textbf{TechNica} desarrolla software y hardware y está evaluando la posibilidad de contratar más personal y adquirir más equipos para aumentar su producción.

\textbf{Datos:}
\begin{itemize}
    \item Personal actual: 50 empleados (costo adicional de 800 dólares por empleado al mes).
    \item Equipos actuales: 30 unidades (costo adicional de 1500 dólares por equipo).
    \item Proyecto de Software \( x_1 \) requiere 2 empleados y 1 equipo por unidad.
    \item Proyecto de Hardware \( x_2 \) requiere 3 empleados y 2 equipos por unidad.
    \item Ingreso por unidad del proyecto de Software: 5000 dólares.
    \item Ingreso por unidad del proyecto de Hardware: 8000 dólares.
\end{itemize}

\textbf{Formulación del Primal:}
\[
\max \quad 5000x_1 + 8000x_2
\]
Sujeto a:
\[
\begin{aligned}
    2x_1 + 3x_2 &\leq 50 \\
    x_1 + 2x_2 &\leq 30 \\
    x_1, x_2 &\geq 0
\end{aligned}
\]

\textbf{Pregunta:}
\begin{enumerate}
    \item Decida si es rentable contratar más personal y adquirir más equipos para maximizar las ganancias.
    \item Analice el significado económico de las variables duales. ¿Cómo influye un precio sombra positivo en las decisiones de expansión de recursos humanos y tecnológicos?
\end{enumerate}

\end{problem}

\end{document}
