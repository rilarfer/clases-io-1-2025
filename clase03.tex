\documentclass{article}
\usepackage{amsmath}

\title{Problemas de Programación Lineal}
\author{Clase de Programación Lineal}
\date{\today}

\begin{document}

\maketitle

\section{Problema 1}
Utilice el método gráfico para resolver el problema:

\[
\text{Maximizar } Z = 10x_1 + 20x_2,
\]

sujeto a

\[
-x_1 + 2x_2 \leq 15
\]
\[
x_1 + x_2 \leq 12
\]
\[
5x_1 + 3x_2 \leq 45
\]

y

\[
x_1 \geq 0, \quad x_2 \geq 0.
\]

\section{Problema 2}
La compañía de seguros Primo está en proceso de introducir dos nuevas líneas de productos: seguro de riesgo especial e hipotecas. La ganancia esperada es de \$5 por el seguro de riesgo especial y de \$2 por unidad de hipoteca. La administración desea establecer las cuotas de venta de las nuevas líneas para maximizar la ganancia total esperada. Los requerimientos de trabajo son los siguientes:

\subsection{Requerimientos de Trabajo}
\begin{tabular}{|c|c|c|c|}
\hline
Departamento & Horas de trabajo por unidad & & Horas de trabajo disponibles \\
 & Riesgo especial & Hipoteca & \\
\hline
Suscripciones & 3 & 2 & 2400 \\
\hline
Administración & 0 & 1 & 800 \\
\hline
Reclamaciones & 2 & 0 & 1200 \\
\hline
\end{tabular}

Formule un modelo de programación lineal y utilice el método gráfico para resolver el modelo.

\section{Problema 3}
Una empresa de software está desarrollando dos tipos de aplicaciones móviles. La aplicación tipo 1 requiere el doble de tiempo de desarrollo que la aplicación tipo 2. Si todo el tiempo de desarrollo disponible se dedica únicamente a la aplicación tipo 2, la empresa puede desarrollar un total de 400 aplicaciones tipo 2 al día. Los límites de mercado respectivos para la aplicación tipo 1 y la aplicación tipo 2 son de 150 y 200 aplicaciones por día, respectivamente. La ganancia es de \$8 por aplicación tipo 1, y de \$5 por aplicación tipo 2. Determine la cantidad de aplicaciones de cada tipo que maximice la ganancia total.

\end{document}
