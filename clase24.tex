\documentclass{article}
\usepackage{amsmath}
\usepackage{amssymb}
\usepackage{geometry}
\usepackage{xcolor}
\usepackage{mdframed}
\geometry{a4paper, margin=1in}

\title{Solución Básica Factible en Transporte}
\author{Ricardo Largaespada}
\date{5 de noviembre 2024}

\newmdenv[
  backgroundcolor=blue!5,
  linecolor=blue,
  linewidth=1pt,
  roundcorner=5pt,
  skipabove=\baselineskip,
  skipbelow=\baselineskip
]{problem}

\begin{document}

\maketitle

\vspace{-1cm}
\begin{problem}
Una empresa tiene tres fábricas (F1, F2, F3) y cuatro almacenes (A1, A2, A3, A4). La capacidad de cada fábrica y la demanda de cada almacén se muestran en la siguiente tabla:

\[
\begin{array}{|c|c|c|c|c|c|}
\hline
           & \text{A1} & \text{A2} & \text{A3} & \text{A4} & \text{Capacidad} \\
\hline
\text{F1}  & 2  & 3  & 11 & 7  & 50        \\
\text{F2}  & 1  & 4  & 2  & 8  & 60        \\
\text{F3}  & 7  & 5  & 9  & 3  & 25        \\
\hline
\text{Demanda} & 30 & 20 & 70 & 15 &           \\
\hline
\end{array}
\]

Calcula la solución inicial básica factible utilizando:
\begin{itemize}
    \item Método de la Esquina Noroeste.
    \item Método del Costo Mínimo.
    \item Método de Aproximación de Vogel.
\end{itemize}
\end{problem}

\vspace{-.5cm}

\begin{problem}
Una compañía produce un producto en dos plantas (P1, P2) y lo envía a tres centros de distribución (C1, C2, C3). Los costos de transporte, las capacidades de las plantas y la demanda de los centros de distribución se presentan a continuación:

\[
\begin{array}{|c|c|c|c|c|}
\hline
           & \text{C1} & \text{C2} & \text{C3} & \text{Capacidad} \\
\hline
\text{P1}  & 4  & 6  & 8  & 40        \\
\text{P2}  & 2  & 5  & 7  & 50        \\
\hline
\text{Demanda} & 30 & 40 & 20 &           \\
\hline
\end{array}
\]

Calcula la solución inicial básica factible para este problema de transporte utilizando:
\begin{itemize}
    \item Método de la Esquina Noroeste.
    \item Método del Costo Mínimo.
    \item Método de Aproximación de Vogel.
\end{itemize}
\end{problem}

\vspace{-.5cm}

\begin{problem}
Una cadena de supermercados cuenta con tres centros de distribución (CD1, CD2, CD3) y necesita abastecer a cinco tiendas (T1, T2, T3, T4, T5). La capacidad de cada centro y la demanda de cada tienda se muestra en la tabla, junto con los costos de transporte.

\[
\begin{array}{|c|c|c|c|c|c|c|}
\hline
           & \text{T1} & \text{T2} & \text{T3} & \text{T4} & \text{T5} & \text{Capacidad} \\
\hline
\text{CD1} & 6  & 4  & 8  & 6  & 3  & 70        \\
\text{CD2} & 3  & 8  & 5  & 9  & 4  & 55        \\
\text{CD3} & 7  & 6  & 4  & 7  & 8  & 60        \\
\hline
\text{Demanda} & 30 & 20 & 50 & 40 & 45 &           \\
\hline
\end{array}
\]

Calcula la solución inicial básica factible utilizando:
\begin{itemize}
    \item Método de la Esquina Noroeste.
    \item Método del Costo Mínimo.
    \item Método de Aproximación de Vogel.
\end{itemize}
\end{problem}

\end{document}
