\documentclass[11pt]{article}

% --------- Idioma y tipografía ----------
\usepackage[spanish, es-nodecimaldot]{babel}
\usepackage[T1]{fontenc}
\usepackage[utf8]{inputenc}
\usepackage{microtype}
\usepackage{newpxtext,newpxmath} % Palatino + matemáticas
\usepackage[a4paper,margin=1in]{geometry}

% --------- Matemática y utilidades -------
\usepackage{amsmath,mathtools} % (sin amssymb para evitar conflicto)

% --------- Colores, enlaces, refs --------
\usepackage{xcolor}
\definecolor{accent}{RGB}{10,80,160}
\usepackage[
  colorlinks,
  linkcolor=accent,
  urlcolor=accent,
  citecolor=accent
]{hyperref}
\usepackage[nameinlink]{cleveref}

% --------- Encabezados/Pies --------------
\usepackage{fancyhdr}
\pagestyle{fancy}
\fancyhf{}
\lhead{\textit{Problemas de Programación Lineal}}
\rhead{Ricardo Largaespada}
\cfoot{\thepage}
\setlength{\headheight}{14pt} % Soluciona el warning

% --------- Tablas bonitas ----------------
\usepackage{booktabs,tabularx}
\renewcommand{\arraystretch}{1.15}

% --------- Cajas para "Problema" ---------
\usepackage[most]{tcolorbox}
\tcbset{
  boxsep=5pt,
  arc=2mm,
  colback=gray!2!white,
  colframe=accent!60!black,
  boxrule=0.6pt
}
\newtcbtheorem[auto counter]{problema}{Problema}{
  fonttitle=\bfseries,
  coltitle=white
}{prob}

% --- TikZ/pgfplots para gráficas ---
\usepackage{pgfplots}
\pgfplotsset{compat=1.18}

% --- Entorno de solución con fondo distinto al de 'problema' ---
% 'problema' usa colback=gray!2!white; aquí usamos un azul muy claro
\newenvironment{solucion}[1][]%
{\begin{tcolorbox}[title=Solución, breakable,
  colback=accent!6!white, colframe=accent!80!black,
  boxrule=0.6pt, fonttitle=\bfseries, #1]}%
{\end{tcolorbox}}

% --------- Título ------------------------
\title{\textbf{Problemas de Programación Lineal}}
\author{Ricardo Largaespada}
\date{\today}

\begin{document}
\maketitle
\thispagestyle{fancy}

\begin{problema}{Método gráfico}{}
Resuelva mediante el método gráfico el siguiente modelo de \emph{maximización}:
\begin{alignat*}{2}
\text{Maximizar}\quad & Z = 10x_1 + 20x_2 \\
\text{sujeto a}\quad
& -x_1 + 2x_2 &&\le 15,\\
& x_1 + x_2 &&\le 12,\\
& 5x_1 + 3x_2 &&\le 45,\\
& x_1,\; x_2 &&\ge 0.
\end{alignat*}
\end{problema}

\begin{solucion}
Modelo:
\[
\max Z=10x_1+20x_2\quad
\text{s.a.}\;
\begin{cases}
-x_1+2x_2\le 15,\\
x_1+x_2\le 12,\\
5x_1+3x_2\le 45,\\
x_1,x_2\ge 0.
\end{cases}
\]
Vértices factibles (primer cuadrante): 
$(0,0)$, $(0,7.5)$, $(3,9)$, $(4.5,7.5)$, $(9,0)$.
Evaluando $Z$: $0,\,150,\,\mathbf{210},\,195,\,90$ respectivamente.
Óptimo en $\boxed{(x_1,x_2)=(3,9)}$ con $\boxed{Z^*=210}$.

\begin{center}
\begin{tikzpicture}
\begin{axis}[
  axis lines=left, xlabel={$x_1$}, ylabel={$x_2$},
  xmin=0, xmax=12, ymin=0, ymax=16,
  grid=both, width=.95\linewidth, height=8cm,
  legend pos=north east]
  % Restricciones (líneas frontera)
\addplot[thick, blue, domain=0:12, samples=2] {(15 + x)/2};             % -x1+2x2=15
\addlegendentry{$-x_1+2x_2=15$}
\addplot[thick, red,  domain=0:12, samples=2] {12 - x};                  % x1+x2=12
\addlegendentry{$x_1+x_2=12$}
\addplot[thick, green!50!black, domain=0:9, samples=2] {(45 - 5*x)/3};   % 5x1+3x2=45 (x-intercept = 9)
\addlegendentry{$5x_1+3x_2=45$}
  % Región factible (relleno)
  \addplot[fill=accent!30!white, fill opacity=0.25, draw=none]
    coordinates {(0,0) (9,0) (4.5,7.5) (3,9) (0,7.5)} -- cycle;
  % Puntos vértice
  \addplot[only marks, mark=*, mark size=1.8pt] coordinates {(0,0) (0,7.5) (3,9) (4.5,7.5) (9,0)};
  % Punto óptimo y recta de isobeneficio
  \addplot[only marks, mark=*, mark size=2.2pt, black] coordinates {(3,9)};
  \node[above right] at (axis cs:3,9) {$(3,9)$};
  \node[above right] at (axis cs:4.5,7.5) {$(4.5,7.5)$};
  \node[above right] at (axis cs:9,0) {$(9,0)$};
  \node[above right] at (axis cs:3,9) {$(3,9)$};
  \node[above right] at (axis cs:0,0) {$(0,0)$};
  \node[above right] at (axis cs:0,7.5) {$(0,7.5)$};
  \addplot[thick, gray!70, dash dot,domain=0:12] {10.5 - 0.5*x}; % 10x1+20x2=210
\end{axis}
\end{tikzpicture}
\end{center}
\end{solucion}

\begin{problema}{Cuotas de venta (Primo Seguros)}{}
La compañía de seguros \emph{Primo} planea introducir dos líneas: riesgo especial y hipotecas.
La ganancia esperada es de \$5 por unidad de riesgo especial y \$2 por hipoteca.
Formule un modelo de programación lineal que \textbf{maximice} la ganancia total y resuélvalo por el método gráfico.

\begin{center}
\begin{tabularx}{\textwidth}{@{}l *{2}{>{\centering\arraybackslash}X} >{\centering\arraybackslash}X @{}}
\toprule
\textbf{Departamento} & \textbf{Riesgo especial (h/u)} & \textbf{Hipoteca (h/u)} & \textbf{Disponibles (h)}\\
\midrule
Suscripciones & 3 & 2 & 2400\\
Administración & 0 & 1 & 800\\
Reclamaciones & 2 & 0 & 1200\\
\bottomrule
\end{tabularx}
\end{center}
\end{problema}

\begin{solucion}
Variables: $R$ (riesgo especial), $H$ (hipotecas).
\[
\max Z=5R+2H\quad
\text{s.a.}\;
\begin{cases}
3R+2H\le 2400 & (\text{Suscripciones})\\
H\le 800      & (\text{Administración})\\
R\le 600      & (\text{Reclamaciones})\\
R,H\ge 0.
\end{cases}
\]
Vértices: $(0,0)$, $(0,800)$, $(\tfrac{800}{3},800)$, $(600,300)$, $(600,0)$.
Valores de $Z$: $0,\,1600,\,\frac{8800}{3}\!\approx\!2933.33,\,\mathbf{3600},\,3000$.
Óptimo en $\boxed{(R,H)=(600,300)}$ con $\boxed{Z^*=3600}$.

\begin{center}
\begin{tikzpicture}
\begin{axis}[
  axis lines=left, xlabel={$R$}, ylabel={$H$},
  xmin=0, xmax=650, ymin=0, ymax=900,
  grid=both, width=.95\linewidth, height=8cm,
  legend pos=north east]
  % Líneas frontera
  \addplot[thick, blue,domain=0:650] {1200 - 1.5*x};     \addlegendentry{$3R+2H=2400$}
  \addplot[thick, red, domain=0:650] {800};       \addlegendentry{$H=800$}
  \addplot[thick, green!50!black, domain=0:650] coordinates {(600,0) (600,900)}; \addlegendentry{$R=600$}
  % Región factible
  \addplot[fill=accent!30!white, fill opacity=0.25, draw=none]
    coordinates {(0,0) (0,800) (266.6667,800) (600,300) (600,0)} -- cycle;
  % Vértices
  \addplot[only marks, mark=*, mark size=1.8pt] 
    coordinates {(0,800) (266.6667,800) (600,300) (600,0)};
  % Óptimo + isoganancia
  \addplot[only marks, mark=*, mark size=2.2pt, black] coordinates {(600,300)};
  \node[above left] at (axis cs:600,300) {$(600,300)$};
  \addplot[thick, gray!70, dash dot,domain=0:650] {(3600 - 5*x)/2}; % 5R+2H=3600
\end{axis}
\end{tikzpicture}
\end{center}
\end{solucion}

\begin{problema}{Desarrollo de aplicaciones móviles}{}
Una empresa desarrolla dos tipos de aplicaciones móviles.
La aplicación tipo 1 requiere el doble de tiempo de desarrollo que la tipo 2.
Si todo el tiempo disponible se dedica a la tipo 2, se pueden producir 400 aplicaciones tipo 2 al día.
Los límites de mercado para tipo 1 y tipo 2 son 150 y 200 por día, respectivamente.
Las ganancias son \$8 (tipo 1) y \$5 (tipo 2).
Determine las cantidades de cada tipo que \textbf{maximizan} la ganancia total.
\end{problema}

\begin{solucion}
Variables: $x_1$ (tipo 1), $x_2$ (tipo 2).  
Tiempo: $2x_1+x_2\le 400$; límites de mercado: $x_1\le 150,\ x_2\le 200$; $x_1,x_2\ge0$.
Vértices: $(0,200)$, $(150,0)$, $(150,100)$, $(100,200)$.
Ganancias: $1000,\,1200,\,1700,\,\mathbf{1800}$.
Óptimo en $\boxed{(x_1,x_2)=(100,200)}$ con $\boxed{Z^*=1800}$.

\begin{center}
\begin{tikzpicture}
\begin{axis}[
  axis lines=left, xlabel={$x_1$}, ylabel={$x_2$},
  xmin=0, xmax=180, ymin=0, ymax=220,
  grid=both, width=.95\linewidth, height=8cm,
  legend pos=north east]
  % Líneas frontera
  \addplot[thick, blue,domain=0:180] {400 - 2*x};        \addlegendentry{$2x_1+x_2=400$}
  \addplot[thick, red, domain=0:180] {200};       \addlegendentry{$x_2=200$}
  \addplot[thick, green!50!black, domain=0:180] coordinates {(150,0) (150,220)}; \addlegendentry{$x_1=150$}
  % Región factible
  \addplot[fill=accent!30!white, fill opacity=0.25, draw=none]
    coordinates {(0,0) (150,0) (150,100) (100,200) (0,200)} -- cycle;
  % Vértices
  \addplot[only marks, mark=*, mark size=1.8pt] 
    coordinates {(0,200) (150,0) (150,100) (100,200)};
  % Óptimo + isoganancia
  \addplot[only marks, mark=*, mark size=2.2pt, black] coordinates {(100,200)};
  \node[above right] at (axis cs:100,200) {$(100,200)$};
  \addplot[thick, gray!70, dash dot,domain=0:180] {(1800 - 8*x)/5}; % 8x1+5x2=1800
\end{axis}
\end{tikzpicture}
\end{center}
\end{solucion}

\end{document}
