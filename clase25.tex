\documentclass{article}
\usepackage{amsmath}
\usepackage{amssymb}
\usepackage{geometry}
\usepackage{xcolor}
\usepackage{mdframed}
\geometry{a4paper, margin=1in}
\usepackage{comment}

\title{Problema de Transporte para Explicación}
\author{Ricardo Largaespada}
\date{\today}

\newmdenv[
  backgroundcolor=blue!5,
  linecolor=blue,
  linewidth=1pt,
  roundcorner=5pt,
  skipabove=\baselineskip,
  skipbelow=\baselineskip
]{problem}

\begin{document}

\maketitle

\vspace{-1cm}

\begin{problem}
\textbf{Problema: Distribución de Suministros desde Almacenes a Centros de Distribución}

Una empresa cuenta con cuatro almacenes (A1, A2, A3, A4) que deben abastecer a cinco centros de distribución (C1, C2, C3, C4, C5). Cada almacén tiene una capacidad de suministro, y cada centro de distribución tiene una demanda específica. Los costos en la tabla a continuación representan el costo de transporte por unidad (en dólares) desde cada almacén hacia cada centro de distribución.

\[
\begin{array}{|c|c|c|c|c|c|c|}
\hline
           & \text{C1} & \text{C2} & \text{C3} & \text{C4} & \text{C5} & \text{Capacidad} \\
\hline
\text{A1}  & 4         & 6         & 8         & 10        & 9         & 30        \\
\text{A2}  & 5         & 3         & 7         & 8         & 6         & 40        \\
\text{A3}  & 6         & 4         & 3         & 5         & 7         & 50        \\
\text{A4}  & 8         & 6         & 4         & 3         & 5         & 30        \\
\hline
\text{Demanda} & 20    & 30        & 25        & 40        & 35        &           \\
\hline
\end{array}
\]

\begin{itemize}
    \item \textbf{Objetivo}: Minimizar el costo total de transporte mientras se cumple con la capacidad de los almacenes y la demanda de los centros de distribución.
\end{itemize}

\end{problem}

\vspace{-0.5cm}
\begin{comment}
\begin{problem}
\textbf{Solución Inicial}

Para obtener una solución inicial, puedes utilizar cualquiera de los métodos, como el \textbf{Método de la Esquina Noroeste}, el \textbf{Método de Costo Mínimo} o el \textbf{Método de Aproximación de Vogel}.

\begin{itemize}
    \item \textbf{Ejemplo: Método de la Esquina Noroeste}
        \begin{enumerate}
            \item Asigna 20 unidades de A1 a C1 (costo: $20 \times 4 = 80$).
            \item Asigna 10 unidades de A1 a C2 (costo: $10 \times 6 = 60$).
            \item Asigna 20 unidades de A2 a C2 (costo: $20 \times 3 = 60$).
            \item Asigna 25 unidades de A2 a C3 (costo: $25 \times 7 = 175$).
            \item Asigna 20 unidades de A3 a C4 (costo: $20 \times 5 = 100$).
            \item Asigna 20 unidades de A3 a C5 (costo: $20 \times 7 = 140$).
            \item Asigna 20 unidades de A4 a C4 (costo: $20 \times 3 = 60$).
            \item Asigna 15 unidades de A4 a C5 (costo: $15 \times 5 = 75$).
        \end{enumerate}
    \item \textbf{Costo Total Inicial}:
        \[
        80 + 60 + 60 + 175 + 100 + 140 + 60 + 75 = 750 \text{ dólares}
        \]
\end{itemize}

\end{problem}

\vspace{-0.5cm}

\begin{problem}
\textbf{Optimización}

1. \textbf{Método de Distribución de Costos Modificados (MODI)}: 
   \begin{itemize}
       \item Calcula los valores de \( u_i \) y \( v_j \) para cada fila y columna, respectivamente.
       \item Determina los valores \( c_{ij} - u_i - v_j \) para las celdas no asignadas.
       \item Identifica si existen valores negativos, lo que indicaría una posible mejora en la solución.
   \end{itemize}

\end{problem}
\end{comment}
\end{document}
