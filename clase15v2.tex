\documentclass{article}
\usepackage{amsmath}
\usepackage{amssymb}
\usepackage{geometry}
\usepackage{xcolor}
\usepackage{mdframed}
\geometry{a4paper, margin=1in}
\usepackage{multicol}

\title{Problemas sobre Dualidad en Programación Lineal}
\author{Ricardo Largaespada}
\date{15 de octubre 2024}

\newmdenv[
  backgroundcolor=blue!5,
  linecolor=blue,
  linewidth=1pt,
  roundcorner=5pt,
  skipabove=\baselineskip,
  skipbelow=\baselineskip
]{problem}

\begin{document}

\maketitle

\section*{Problema 1}

\textbf{Datos:}

\begin{multicols}{2}
\begin{itemize}
    \item Horas de desarrollo disponibles: 200 horas.
    \item Espacio en servidores disponible: 150 GB.
    \item \textbf{Sistema Operativo:}
    \begin{itemize}
        \item Requiere 50 horas de desarrollo por unidad.
        \item Requiere 30 GB de espacio por unidad.
        \item Ganancia: \$200 por unidad.
    \end{itemize}
    \item \textbf{Aplicación Móvil:}
    \begin{itemize}
        \item Requiere 40 horas de desarrollo por unidad.
        \item Requiere 20 GB de espacio por unidad.
        \item Ganancia: \$150 por unidad.
    \end{itemize}
\end{itemize}

\textbf{Variables de decisión:}

\[
\begin{aligned}
x_1 &= \text{Cantidad de Sistemas Operativos producidos} \\
x_2 &= \text{Cantidad de Aplicaciones Móviles producidas}
\end{aligned}
\]
\end{multicols}

\begin{multicols}{2}
\textbf{Función objetivo:}

\[
\max Z = 200x_1 + 150x_2
\]

\textbf{Sujeto a las restricciones:}

\[
\begin{aligned}
50x_1 + 40x_2 &\leq 200 & \text{(Horas de desarrollo)} \\
30x_1 + 20x_2 &\leq 150 & \text{(Espacio en servidores)} \\
x_1, x_2 &\geq 0 & \text{(No negatividad)}
\end{aligned}
\]

\textbf{Formulación del problema dual:}

Sea \( y_1 \) y \( y_2 \) las variables duales asociadas a las restricciones.

\textbf{Función objetivo dual:}

\[
\min W = 200y_1 + 150y_2
\]

\textbf{Sujeto a:}

\[
\begin{aligned}
50y_1 + 30y_2 &\geq 200 \\
40y_1 + 20y_2 &\geq 150 \\
y_1, y_2 &\geq 0
\end{aligned}
\]
    
\end{multicols}

\textbf{Resolución del problema dual:}

Resolvemos el sistema de ecuaciones:

Multiplicamos la segunda restricción dual por 1.5:

\[
1.5(40y_1 + 20y_2) \geq 1.5(150) \implies 60y_1 + 30y_2 \geq 225
\]

Restamos la primera restricción dual de esta nueva ecuación:

\[
(60y_1 + 30y_2) - (50y_1 + 30y_2) \geq 225 - 200 \implies 10y_1 \geq 25 \implies y_1 \geq 2.5
\]

Sustituyendo \( y_1 = 2.5 \) en la primera restricción dual:

\[
50(2.5) + 30y_2 \geq 200 \implies 125 + 30y_2 \geq 200 \implies 30y_2 \geq 75 \implies y_2 \geq 2.5
\]

\textbf{Precios sombra:}

\[
\begin{aligned}
y_1 &= \$2.5 \text{ por hora de desarrollo} \\
y_2 &= \$2.5 \text{ por GB de espacio en servidores}
\end{aligned}
\]

\textbf{Interpretación económica:}

Los precios sombra indican que por cada unidad adicional de recurso (hora de desarrollo o GB de espacio), la ganancia máxima aumentaría en \$2.5.

\textbf{Recomendación:}

La empresa debería adquirir más horas de desarrollo y espacio en servidores para maximizar sus beneficios, ya que ambos recursos tienen precios sombra positivos.

\newpage

\section*{Problema 2}

\begin{multicols}{2}
\textbf{Datos:}

\begin{itemize}
    \item Ancho de banda disponible: 400 Mbps.
    \item Antenas disponibles: 20 unidades.
    \item \textbf{Plan Básico} (\( x_1 \)):
    \begin{itemize}
        \item Requiere 1 Mbps y 0.5 antenas por usuario.
        \item Ingreso por usuario: \$30.
    \end{itemize}
    \item \textbf{Plan Premium} (\( x_2 \)):
    \begin{itemize}
        \item Requiere 3 Mbps y 1 antena por usuario.
        \item Ingreso por usuario: \$70.
    \end{itemize}
\end{itemize}

\textbf{Variables de decisión:}

\[
\begin{aligned}
x_1 &= \text{Número de usuarios del Plan Básico} \\
x_2 &= \text{Número de usuarios del Plan Premium}
\end{aligned}
\]
    
\end{multicols}

\begin{multicols}{2}
\textbf{Función objetivo:}

\[
\max Z = 30x_1 + 70x_2
\]

\textbf{Sujeto a las restricciones:}

\[
\begin{aligned}
1x_1 + 3x_2 &\leq 400 & \text{(Ancho de banda)} \\
0.5x_1 + 1x_2 &\leq 20 & \text{(Antenas)} \\
x_1, x_2 &\geq 0 & \text{(No negatividad)}
\end{aligned}
\]

\textbf{Formulación del problema dual:}

Sea \( y_1 \) y \( y_2 \) las variables duales.

\textbf{Función objetivo dual:}

\[
\min W = 400y_1 + 20y_2
\]

\textbf{Sujeto a:}

\[
\begin{aligned}
1y_1 + 0.5y_2 &\geq 30 \\
3y_1 + 1y_2 &\geq 70 \\
y_1, y_2 &\geq 0
\end{aligned}
\]
    
\end{multicols}

\textbf{Resolución del problema dual:}

Resolvemos el sistema de ecuaciones:

Multiplicamos la primera restricción dual por 2:

\[
2(1y_1 + 0.5y_2) \geq 2(30) \implies 2y_1 + y_2 \geq 60
\]

Restamos esta ecuación de la segunda restricción dual:

\[
(3y_1 + y_2) - (2y_1 + y_2) \geq 70 - 60 \implies y_1 \geq 10
\]

Sustituyendo \( y_1 = 10 \) en la primera restricción dual:

\[
1(10) + 0.5y_2 \geq 30 \implies 10 + 0.5y_2 \geq 30 \implies 0.5y_2 \geq 20 \implies y_2 \geq 40
\]

\textbf{Precios sombra:}

\[
\begin{aligned}
y_1 &= \$10 \text{ por Mbps de ancho de banda} \\
y_2 &= \$40 \text{ por antena}
\end{aligned}
\]

\textbf{Análisis de rentabilidad:}

- Costo adicional por Mbps: \$250.
- Precio sombra del Mbps: \$10.

Como el costo es mayor que el precio sombra (\$250 > \$10), no es rentable invertir en ancho de banda adicional.

- Costo adicional por antena: \$2,000.
- Precio sombra de la antena: \$40.

Como el costo es mayor que el precio sombra (\$2,000 > \$40), no es rentable invertir en antenas adicionales.

\textbf{Conclusión:}

No es rentable invertir en capacidad adicional de ancho de banda ni en antenas, ya que los costos superan a los precios sombra.

\newpage

\section*{Problema 3}

\begin{multicols}{2}
    
\textbf{Datos:}

\begin{itemize}
    \item Personal disponible: 50 empleados.
    \item Equipos disponibles: 30 unidades.
    \item \textbf{Proyecto de Software} (\( x_1 \)):
    \begin{itemize}
        \item Requiere 2 empleados y 1 equipo por unidad.
        \item Ingreso por unidad: \$5,000.
    \end{itemize}
    \item \textbf{Proyecto de Hardware} (\( x_2 \)):
    \begin{itemize}
        \item Requiere 3 empleados y 2 equipos por unidad.
        \item Ingreso por unidad: \$8,000.
    \end{itemize}
\end{itemize}

\textbf{Variables de decisión:}

\[
\begin{aligned}
x_1 &= \text{Cantidad de proyectos de Software} \\
x_2 &= \text{Cantidad de proyectos de Hardware}
\end{aligned}
\]
\end{multicols}

\begin{multicols}{2}
\textbf{Función objetivo:}

\[
\max Z = 5,000x_1 + 8,000x_2
\]

\textbf{Sujeto a las restricciones:}

\[
\begin{aligned}
2x_1 + 3x_2 &\leq 50 & \text{(Personal)} \\
1x_1 + 2x_2 &\leq 30 & \text{(Equipos)} \\
x_1, x_2 &\geq 0 & \text{(No negatividad)}
\end{aligned}
\]

\textbf{Formulación del problema dual:}

Sea \( y_1 \) y \( y_2 \) las variables duales.

\textbf{Función objetivo dual:}

\[
\min W = 50y_1 + 30y_2
\]

\textbf{Sujeto a:}

\[
\begin{aligned}
2y_1 + 1y_2 &\geq 5,000 \\
3y_1 + 2y_2 &\geq 8,000 \\
y_1, y_2 &\geq 0
\end{aligned}
\]
\end{multicols}

\textbf{Resolución del problema dual:}

Multiplicamos la primera restricción dual por 2:

\[
2(2y_1 + 1y_2) \geq 2(5,000) \implies 4y_1 + 2y_2 \geq 10,000
\]

Restamos la segunda restricción dual de esta ecuación:

\[
(4y_1 + 2y_2) - (3y_1 + 2y_2) \geq 10,000 - 8,000 \implies y_1 \geq 2,000
\]

Sustituyendo \( y_1 = 2,000 \) en la primera restricción dual:

\[
2(2,000) + y_2 \geq 5,000 \implies 4,000 + y_2 \geq 5,000 \implies y_2 \geq 1,000
\]

\textbf{Precios sombra:}

\[
\begin{aligned}
y_1 &= \$2,000 \text{ por empleado} \\
y_2 &= \$1,000 \text{ por equipo}
\end{aligned}
\]

\textbf{Análisis de rentabilidad:}

- Costo adicional por empleado: \$800.
- Precio sombra del empleado: \$2,000.

Como el precio sombra es mayor que el costo (\$2,000 > \$800), es rentable contratar más personal.

- Costo adicional por equipo: \$1,500.
- Precio sombra del equipo: \$1,000.

Como el costo es mayor que el precio sombra (\$1,500 > \$1,000), no es rentable adquirir más equipos.

\textbf{Conclusión:}

La empresa debería contratar más personal para maximizar sus ganancias, pero no es rentable adquirir más equipos.

\end{document}
