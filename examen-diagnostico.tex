\documentclass[11pt,paper=a4,answers, addpoints]{exam}
\usepackage{graphicx,lastpage,comment}
\usepackage{upgreek}
\usepackage{censor}
\censorruledepth=-.2ex
\censorruleheight=.1ex
\hyphenpenalty 10000
\usepackage[paperheight=10.5in,paperwidth=8.27in,bindingoffset=0in,left=0.8in,right=1in, top=1.3in,bottom=1in,headsep=.5\baselineskip]{geometry}
\flushbottom
\usepackage[normalem]{ulem}
\usepackage[utf8]{inputenc}
\usepackage[spanish]{babel}
\renewcommand\ULthickness{2pt}
\setlength\ULdepth{1.5ex}
\renewcommand{\baselinestretch}{1}
\pagestyle{empty}

\pagestyle{headandfoot}
\headrule
\newcommand{\continuedmessage}{%
  \ifcontinuation{\footnotesize Pregunta \ContinuedQuestion\ continua\ldots}{}%
}
\runningheader{\footnotesize Programa de Ingeniería de Sistemas}
{\footnotesize Prueba Diagnóstica — Día 1}
{\footnotesize Página \thepage\ de \numpages}
\footrule
\footer{\footnotesize }
{}
{\ifincomplete{\footnotesize Pregunta \IncompleteQuestion\ continua en la siguiente página \ldots}
  {\iflastpage{\footnotesize Final del examen}{\footnotesize Por favor vea la siguiente página\ldots}}}
\usepackage{amsfonts,amsmath}
\usepackage{cleveref}
\crefname{figure}{figura}{figuras}
\crefname{question}{pregunta}{preguntas}
\renewcommand\thequestion{\arabic{question}}
\renewcommand{\questionlabel}{\thequestion)}
\renewcommand{\questionshook}{%
  \setlength{\leftmargin}{0pt}%
  \setlength{\labelwidth}{-\labelsep}%
}
\decimalpoint
\nopointsinmargin
\pointpoints{Punto}{Punto}
\marginpointname{\points}
\pointformat{\boldmath\themarginpoints}
\bracketedpoints
\usepackage{quoting,xparse}
\pointpoints{punto}{puntos}
\bonuspointpoints{punto extra}{puntos extra}
\totalformat{Pregunta \thequestion: \totalpoints puntos}
\hqword{Pregunta}
\hpgword{Página}
\hpword{Puntos}
\hsword{Puntos obtenidos}
\htword{Total}
\usepackage{circuitikz}
\usepackage{color}
\usepackage{pgfplots,graphicx}
\pgfplotsset{compat=1.18}

\usepackage{mathrsfs}
\newcommand{\Laplace}[1]{\ensuremath{\mathscr{L}{\left\lbrace #1\right\rbrace}}}
\newcommand{\InvLap}[1]{\ensuremath{\mathscr{L}^{-1}{\left\lbrace #1\right\rbrace}}}

\usepackage{background}
\backgroundsetup{
  scale=1,
  color=black,
  opacity=1,
  angle=0,
  pages=all,
  position=current page.south west,
  nodeanchor=south west,
  vshift=0mm,
  hshift=0mm,
  contents={\includegraphics[width=\paperwidth,height=\paperheight]{fondo.pdf}}
}

\begin{document}
\printanswers
%\noprintanswers
\shorthandoff{<>}
\thispagestyle{empty}

%==================== CABECERA ====================
\begin{center}
    \textit{\textbf{Prueba Diagnóstica — Día 1}}
\end{center}
\vspace{-.1cm}
\noindent
\begin{minipage}[t]{.6\textwidth}%
  {\bfseries Nombres}: \makebox[.75\textwidth]{\hrulefill} \par
  {\bfseries Apellidos}: \makebox[.75\textwidth]{\hrulefill} \par
  {\bfseries Docente}: Ricardo Largaespada
\end{minipage}%
\hfill
\begin{minipage}[t]{.4\textwidth}%
  {\bfseries Curso}: investigación de Operaciones I \par
  {\bfseries Fecha}: 11 de agosto de 2025 \par
  {\bfseries Grupo:} \rule{2.8cm}{.4pt}
\end{minipage}
\par\noindent\rule{\textwidth}{1pt}

%==================== INICIO PREGUNTAS ====================
\begin{questions}

%-------- BLOQUE 1: IO e IO-1 ------------
\titledquestion{Diagnóstico para IO-1}[10]
Resuelva los siguientes casos.

\begin{parts}
  \part[1] Grafique la recta \(5x+3y=25\). Intercepte y rotule ambos ejes.
 \begin{solution}
Intersecciones: al fijar $y=0\Rightarrow x=5,(5,0)$; al fijar $x=0\Rightarrow y=\tfrac{25}{3},\bigl(0,\tfrac{25}{3}\bigr)$. Ecuación: $y=\dfrac{25-5x}{3}$. Gráfica:

{\centering
\begin{tikzpicture}
\begin{axis}[
axis lines=middle, xlabel={$x$}, ylabel={$y$},
xmin=-1, xmax=6, ymin=-1, ymax=10,
grid=both, samples=200, width=.85\linewidth, height=6.2cm]
\addplot[domain=-1:8, thick] {(25 - 5*x)/3};
\addplot[only marks] coordinates {(5,0) (0,25/3)};
\node[above right] at (axis cs:0,25/3) {$\left(0,\tfrac{25}{3}\right)$};
\node[below] at (axis cs:5,0) {$(5,0)$};
\end{axis}
\end{tikzpicture}\par}
\end{solution} 
  \part[3] En un mismo sistema cartesiano, grafique el sistema
  \[
  \begin{aligned}
  40F+50S&=200,\\
  20S&=50,\\
  60F+30S&=210.
  \end{aligned}
  \]
  Use \(F\) en el eje \(x\) y \(S\) en el eje \(y\). Indique el punto de intersección (si existe).
\begin{solution}
Despejando:
\[
\begin{aligned}
40F+50S&=200 &&\Rightarrow&& S=4-0.8F,\\
20S&=50      &&\Rightarrow&& S=2.5,\\
60F+30S&=210 &&\Rightarrow&& S=7-2F.
\end{aligned}
\]

Las dos primeras se cortan en \((F,S)=\left(\tfrac{75}{40},\,\tfrac{5}{2}\right)=(1.875,\,2.5)\).
Al sustituir en la tercera: \(60(1.875)+30(2.5)=187.5\neq210\), por tanto \emph{no hay intersección común de las tres}.

\begin{center}
\begin{tikzpicture}
\begin{axis}[
  axis lines=middle,
  xlabel={$F$}, ylabel={$S$},
  xmin=-1, xmax=5, ymin=-1, ymax=7,
  grid=both, samples=200, width=.9\linewidth, height=7cm,
  legend style={at={(0.98,0.98)},anchor=north east}]
  \addplot[thick, blue] {4 - 0.8*x};
  \addlegendentry{$40F+50S=200$}
  \addplot[thick, red, dashed] {2.5};
  \addlegendentry{$20S=50$}
  \addplot[thick, green!50!black, dotted] {7 - 2*x};
  \addlegendentry{$60F+30S=210$}
  \addplot[only marks, mark=*, mark size=2pt, black] coordinates {(1.875,2.5)};
  \node[above right] at (axis cs:1.875,2.5) {$(1.875,\,2.5)$};
\end{axis}
\end{tikzpicture}
\end{center}
\end{solution}

  \part[3] Resuelva el modelo anterior por el método de Gauss–Jordan (muestre las operaciones elementales).
%\fillwithdottedlines{3.5cm}
\begin{solution}
Matriz aumentada y operaciones elementales:
\[
\left[
\begin{array}{cc|c}
40 & 50 & 200\\
0  & 20 & 50\\
60 & 30 & 210
\end{array}
\right]
\;\xrightarrow{R_2\leftarrow \tfrac{1}{20}R_2}\;
\left[
\begin{array}{cc|c}
40 & 50 & 200\\
0  & 1  & 2.5\\
60 & 30 & 210
\end{array}
\right]
\;\xrightarrow{R_1\leftarrow R_1-50R_2}\;
\left[
\begin{array}{cc|c}
40 & 0 & 75\\
0  & 1 & 2.5\\
60 & 30 & 210
\end{array}
\right]
\]
\[
\xrightarrow{R_1\leftarrow \tfrac{1}{40}R_1}\;
\left[
\begin{array}{cc|c}
1 & 0 & \tfrac{75}{40}\\
0 & 1 & 2.5\\
60 & 30 & 210
\end{array}
\right]
\;\xrightarrow{R_3\leftarrow R_3-30R_2}\;
\left[
\begin{array}{cc|c}
1 & 0 & \tfrac{75}{40}\\
0 & 1 & 2.5\\
60 & 0 & 135
\end{array}
\right]
\;\xrightarrow{R_3\leftarrow R_3-60R_1}\;
\left[
\begin{array}{cc|c}
1 & 0 & \tfrac{75}{40}\\
0 & 1 & 2.5\\
0 & 0 & \tfrac{45}{2}
\end{array}
\right].
\]
Como aparece la fila $[\,0\ \ 0\mid \tfrac{45}{2}\,]$ (contradicción), el sistema de tres ecuaciones es \textbf{inconsistente}.
\end{solution}

  \part[3] Sea
  \[
  A=\begin{pmatrix}
  3&8&0\\
  5&-1&4\\
  1&2&-3
  \end{pmatrix}.
  \]
  \begin{enumerate}
    \item[{\bf a)}] Determine \(A^{-1}\), si existe.
    \item[{\bf b)}] Halle \(A^{\mathsf T}\).
  \end{enumerate}
\begin{solution}
Sea $A=\begin{pmatrix}3&8&0\\[2pt]5&-1&4\\[2pt]1&2&-3\end{pmatrix}$. 
Calculando (por cofactores o Gauss), $\det(A)=137\neq0\Rightarrow A$ es invertible.
Una inversión (vía adjunta o $[A\,|\,I]$) da
\[
A^{-1}=\frac{1}{137}\begin{pmatrix}
-5 & 24 & 32\\[2pt]
19 & -9 & -12\\[2pt]
11 & 2 & -43
\end{pmatrix}.
\]
La transpuesta:
\[
A^{\mathsf T}=
\begin{pmatrix}
3 & 5 & 1\\[2pt]
8 & -1 & 2\\[2pt]
0 & 4 & -3
\end{pmatrix}.
\]
\end{solution}
\end{parts}

% (Opcional, sin puntaje o con bonus)
\bonusquestion[0]{Reflexión (sin puntaje)}
\begin{parts}
  \part ¿Qué desea que \textbf{ocurra} en su aprendizaje en este curso?
\fillwithdottedlines{2cm}
  \part ¿Qué desea que \textbf{no ocurra} en su aprendizaje en este curso?
\fillwithdottedlines{2cm}
  \part ¿Desea trabajar en equipo? Fundamente brevemente.
\fillwithdottedlines{2cm}
  \part ¿Qué está dispuesto(a) a ofrecer para alcanzar los objetivos del curso?
\fillwithdottedlines{2cm}
\end{parts}

\end{questions}

\end{document}
