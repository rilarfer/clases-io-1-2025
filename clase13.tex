\documentclass{beamer}

\usepackage[utf8]{inputenc}
\usepackage[spanish]{babel}
\usepackage{amsmath}
\usepackage[nosetup]{evan}
\usetheme{Madrid}
%\usetheme{Goddard}
\hypersetup{colorlinks,allcolors=.,urlcolor=magenta}
\usepackage[table]{xcolor} % Para definir colores en tablas
\usepackage{graphicx} % Para redimensionar la tabla

\title{Investigación de Operaciones I}
\subtitle{Unidad 2: Aspectos de Dualidad y Análisis de Sensibilidad}
\author[Ricardo Largaespada]{Ricardo Jesús Largaespada Fernández}
\institute[UNI]{Ingeniería de Sistemas, DACTIC, UNI}
\date{07 de Octubre, 2024}

\begin{document}

\frame{\titlepage}

\begin{frame}
\frametitle{Agenda}
\tableofcontents
\end{frame}

\begin{frame}
\frametitle{Objetivos de la Unidad 2}
\begin{itemize}
    \item \textbf{Conceptual}: Relacionar un problema de Programación Lineal con su respectivo Problema Dual, comparando las relaciones entre ambos, que permita la determinación de la solución de uno de ellos a partir del otro.
    \item \textbf{Procedimental}: Formular planteamientos post óptimos a un problema de Programación Lineal, realizando el análisis de sensibilidad y obteniendo soluciones a ejecutarse en una empresa a mediano y largo plazo con la mayor eficiencia y confiabilidad posible.
    \item \textbf{Actitudinal}: Valorar los resultados obtenidos del análisis de sensibilidad, proponiendo distintas formas de mejorar las actividades de una empresa en forma eficiente y científica.
\end{itemize}
\end{frame}

\section{Dual de un Problema Lineal}
\begin{frame}
\frametitle{Dual de un programa Lineal.}
\begin{itemize}
\item Obtención del dual.\\
\item Relaciones Primal - Dual
\end{itemize}
\pause
\textbf{Objetivos de Aprendizaje}
    \begin{enumerate}
        \item Comprender el concepto de dualidad en programación lineal y ser capaz de identificar el dual de un programa lineal dado.
        \item Analizar las relaciones entre el programa primal y su dual, identificando cómo los cambios en uno afectan al otro y viceversa.
    \end{enumerate}
\end{frame}

\begin{frame}{Límites Superiores de un Problema de Maximización}
    \textbf{Considera el siguiente PL:}
    \[
    z^* = \max \quad 4x_1 + 5x_2 + 8x_3 \\
    \text{s.a.} \quad x_1 + 2x_2 + 3x_3 \leq 6 \\
    2x_1 + x_2 + 2x_3 \leq 4 \\
    x_1 \geq 0, \quad x_2 \geq 0, \quad x_3 \geq 0.
    \]
    
    \vspace{0.3cm}
    \textbf{Supón que el PL es muy difícil de resolver.}
    \begin{itemize}
        \item Tu amigo propone una solución \(\hat{x} = \left(\frac{1}{2}, 1, 1\right)\) con \(\hat{z} = 15\).
        \item Si conocemos \( z^* \), podemos comparar \(\hat{z}\) con \( z^* \).
        \item ¿Cómo evaluar el rendimiento de \(\hat{x}\) \textbf{\textcolor{blue}{sin}} resolver el PL?
    \end{itemize}
    
    \vspace{0.3cm}
    \textbf{Si encontramos un \textcolor{blue}{límite superior} de \( z^* \), eso funcionaría!}
    \begin{itemize}
        \item \( z^* \) no puede ser mayor que el límite superior.
        \item Entonces, si \(\hat{z}\) está cerca del límite superior, \(\hat{x}\) es bastante bueno.
    \end{itemize}

\end{frame}

\begin{frame}{Límites Superiores de un Problema de Maximización}
    \textbf{¿Cómo encontrar un límite superior de \( z^* \) para el siguiente problema?}
    \[
    z^* = \max \quad 4x_1 + 5x_2 + 8x_3 \\
    \text{s.a.} \quad x_1 + 2x_2 + 3x_3 \leq 6 \\
    2x_1 + x_2 + 2x_3 \leq 4 \\
    x_1 \geq 0, \quad x_2 \geq 0, \quad x_3 \geq 0.
    \]
    
    \vspace{0.3cm}
    \textbf{¿Qué tal esto? Multipliquemos la primera restricción por 2, y la segunda por 1, y luego sumémoslas:}
    \[
    2(x_1 + 2x_2 + 3x_3) + (2x_1 + x_2 + 2x_3) \leq 2 \times 6 + 4 \\
    \Rightarrow 4x_1 + 5x_2 + 8x_3 \leq 16.
    \]

    \vspace{0.3cm}
    \textbf{Compara esto con la función objetivo, sabemos que} \( z^* \leq 16 \).
    \begin{itemize}
        \item Tal vez \( z^* \) sea exactamente 16 (y el límite superior sea \textcolor{blue}{ajustado}). Sin embargo, no lo sabemos con certeza.
        \item \(\hat{z} = 15\) está cerca de \( z^* = 16 \), entonces \(\hat{x}\) es bastante bueno.
    \end{itemize}
\end{frame}

\begin{frame}{Límites Superiores de un Problema de Maximización}
    \textbf{¿Cómo encontrar un límite superior de \( z^* \) para este problema?}
    \[
    z^* = \max \quad \textcolor{blue}{3x_1} + \textcolor{blue}{4x_2} + 8x_3 \\
    \text{s.a.} \quad x_1 + 2x_2 + 3x_3 \leq 6 \\
    2x_1 + x_2 + 2x_3 \leq 4 \\
    x_1 \geq 0, \quad x_2 \geq 0, \quad x_3 \geq 0.
    \]

    \vspace{0.3cm}
    \textbf{16 también es un límite superior:}
    \[
    3x_1 + 4x_2 + 8x_3 \\
    \leq 4x_1 + 5x_2 + 8x_3 \quad \text{(porque \( x_1 \geq 0, \, x_2 \geq 0 \))} \\
    = 2(x_1 + 2x_2 + 3x_3) + (2x_1 + x_2 + 2x_3) \\
    \leq 2 \times 6 + 4 = 16.
    \]

    \vspace{0.3cm}
    \textbf{Es muy probable que 16 no sea un límite superior \textcolor{blue}{ajustado} y que haya uno mejor. ¿Cómo mejorar nuestro límite superior?}
\end{frame}

\begin{frame}{¿Mejores Límites Superiores?}
    \[
    z^* = \max \quad \textcolor{blue}{3x_1} + \textcolor{blue}{4x_2} + 8x_3 \\
    \text{s.a.} \quad x_1 + 2x_2 + 3x_3 \leq 6 \\
    2x_1 + x_2 + 2x_3 \leq 4 \\
    x_1 \geq 0, \quad x_2 \geq 0, \quad x_3 \geq 0.
    \]

    \vspace{0.3cm}
    \textbf{Cambiar los \textcolor{blue}{coeficientes} multiplicados en las dos restricciones modifica el límite superior propuesto.}
    \begin{itemize}
        \item Diferentes coeficientes dan lugar a diferentes \textcolor{blue}{combinaciones lineales}.
    \end{itemize}

    \vspace{0.3cm}
    \textbf{Llamemos a los dos coeficientes \( y_1 \) y \( y_2 \), respectivamente:}
    
    \[
    \begin{array}{cccc}
    x_1 + & 2x_2 + & 3x_3 \leq 6 & \quad (\times y_1) \\
    2x_1 + & x_2 + & 2x_3 \leq 4 & \quad (\times y_2) \\
    \hline
    (y_1 + 2y_2)x_1 + & (2y_1 + y_2)x_2 + & (3y_1 + 2y_2)x_3 \leq 6y_1 + 4y_2
    \end{array}
    \]

    \vspace{0.3cm}
    \textbf{Necesitamos} \( y_1 \geq 0 \) y \( y_2 \geq 0 \) para preservar el “\(\leq\)”.

    \vspace{0.3cm}
    \textbf{¿Cuándo tenemos} \( z^* \leq 6y_1 + 4y_2 \)?
\end{frame}

\begin{frame}{Buscando el Límite Superior Más Bajo}
    \textbf{Entonces buscamos dos variables \( y_1 \) y \( y_2 \) tales que:}
    \begin{itemize}
        \item \( y_1 \geq 0 \) y \( y_2 \geq 0 \).
        \item \( 3 \leq y_1 + 2y_2, \, 4 \leq 2y_1 + y_2, \, 8 \leq 3y_1 + 2y_2 \).
        \item Entonces \( z^* \leq 6y_1 + 4y_2 \).
    \end{itemize}

    \vspace{0.3cm}
    \textbf{Para intentar nuestro \textcolor{blue}{mejor} esfuerzo en encontrar un límite superior, minimizamos \( 6y_1 + 4y_2 \).}
    \begin{itemize}
        \item Estamos resolviendo \textbf{\textcolor{blue}{otro}} problema lineal.
    \end{itemize}

    \vspace{0.3cm}
    \[
    \begin{array}{lll}
    \text{max} & 3x_1 + 4x_2 + 8x_3 & \\
    \text{s.a.} & x_1 + 2x_2 + 3x_3 \leq 6 & \\
    & 2x_1 + x_2 + 2x_3 \leq 4 & \\
    & x_1 \geq 0, \, x_2 \geq 0, \, x_3 \geq 0. &
    \end{array}
    \quad \Rightarrow \quad
    \begin{array}{lll}
    \text{min} & 6y_1 + 4y_2 & \\
    \text{s.a.} & y_1 + 2y_2 \geq 3 & \\
    & 2y_1 + y_2 \geq 4 & \\
    & 3y_1 + 2y_2 \geq 8 & \\
    & y_1 \geq 0, \, y_2 \geq 0. &
    \end{array}
    \]

    \vspace{0.3cm}
    \textbf{Llamamos al problema lineal original el \textcolor{blue}{PL primal} y al nuevo su \textcolor{blue}{PL dual}.}

    \vspace{0.3cm}
    \textbf{Esta idea se aplica a \textcolor{blue}{cualquier} PL. Veamos más ejemplos.}
\end{frame}

\begin{frame}{Variables No Positivas o Variables Libres}
    \textbf{Supongamos que las variables no son todas no negativas:}
    \[
    z^* = \max \quad 3x_1 + \textcolor{blue}{4x_2} + 8x_3 \\
    \text{s.a.} \quad x_1 + 2x_2 + 3x_3 \leq 6 \\
    2x_1 + x_2 + 2x_3 \leq 4 \\
    x_1 \geq 0, \quad x_2 \leq 0, \quad x_3 \quad \text{\textcolor{blue}{libre}}.
    \]

    \vspace{0.3cm}
    \textbf{Si queremos:}
    \[
    3x_1 + \quad \textcolor{blue}{4x_2} + \quad 8x_3 \\
    \leq \quad (y_1 + 2y_2)x_1 \quad + \quad (2y_1 + y_2)x_2 \quad + \quad (3y_1 + 2y_2)x_3,
    \]

    \vspace{0.3cm}
    \textbf{Ahora necesitamos:}
    \[
    \begin{array}{lcl}
    y_1 + 2y_2 & \geq & 3 \quad \text{porque} \quad x_1 \geq 0, \\
    2y_1 + y_2 & \leq & 4 \quad \text{porque} \quad x_2 \leq 0, \\
    3y_1 + 2y_2 & = & 8 \quad \text{porque} \quad x_3 \quad \text{es \textcolor{blue}{libre}}.
    \end{array}
    \]
\end{frame}

\begin{frame}{Variables No Positivas o Variables Libres}
    \textbf{Entonces, los PL primal y dual son:}
    \[
    \begin{array}{lll}
    \text{max} & 3x_1 + \textcolor{blue}{4x_2} + 8x_3 & \\
    \text{s.a.} & x_1 + 2x_2 + 3x_3 \leq 6 & \\
    & 2x_1 + x_2 + 2x_3 \leq 4 & \\
    & x_1 \geq 0, \, x_2 \leq 0, \, x_3 \quad \text{\textcolor{blue}{libre}}. &
    \end{array}
    \quad \text{y} \quad
    \begin{array}{lll}
    \text{min} & 6y_1 + 4y_2 & \\
    \text{s.a.} & y_1 + 2y_2 \geq 3 & \\
    & 2y_1 + y_2 \leq 4 & \\
    & 3y_1 + 2y_2 = 8 & \\
    & y_1 \geq 0, \, y_2 \geq 0. &
    \end{array}
    \]

    \vspace{0.3cm}
    \textbf{Algunas observaciones:}
    \begin{itemize}
        \item \textbf{Primal max} \(\Rightarrow\) \textbf{Dual min}.
        \item \textbf{Objetivo primal} \(\Rightarrow\) \textbf{RHS dual}.
        \item \textbf{RHS primal} \(\Rightarrow\) \textbf{Objetivo dual}.
    \end{itemize}

    \vspace{0.3cm}
    \textbf{Además:}
    \begin{itemize}
        \item Variable primal \(\geq 0\) \(\Rightarrow\) Restricción dual \(\geq\).
        \item Variable primal \(\leq 0\) \(\Rightarrow\) Restricción dual \(\leq\).
        \item Variable libre primal \(\Rightarrow\) Restricción dual \(\textcolor{blue}{=}\).
    \end{itemize}

    \vspace{0.3cm}
    \textbf{¿Qué pasa si tenemos restricciones primal \(\geq\) o \(\textcolor{blue}{=}\)?}
\end{frame}

\begin{frame}{Restricciones de No Menor Que y de Igualdad}
    \textbf{Supongamos que las restricciones no son todas “\(\leq\)”:}
    \[
    z^* = \max \quad 3x_1 + \textcolor{blue}{4x_2} + 8x_3 \\
    \text{s.a.} \quad x_1 + 2x_2 + 3x_3 \quad \text{\textcolor{blue}{$\geq$}} \quad 6 \\
    2x_1 + x_2 + 2x_3 \quad \text{\textcolor{blue}{$=$}} \quad 4 \\
    x_1 \geq 0, \quad x_2 \leq 0, \quad x_3 \quad \text{libre}.
    \]

    \vspace{0.3cm}
    \textbf{Para obtener:}
    \[
    y_1(x_1 + 2x_2 + 3x_3) + y_2(2x_1 + x_2 + 2x_3) \leq 6y_1 + 4y_2,
    \]

    \vspace{0.3cm}
    \textbf{Ahora necesitamos que} \( y_1 \leq 0 \). \( y_2 \) puede tener cualquier signo (es decir, libre).
\end{frame}

\begin{frame}{Restricciones de No Menor Que y de Igualdad}
    \textbf{Entonces, los PL primal y dual son:}
    \[
    \begin{array}{lll}
    \text{max} & 3x_1 + \textcolor{blue}{4x_2} + 8x_3 & \\
    \text{s.a.} & x_1 + 2x_2 + 3x_3 \quad \text{\textcolor{blue}{$\geq$}} \quad 6 & \\
    & 2x_1 + x_2 + 2x_3 \quad \text{\textcolor{blue}{$=$}} \quad 4 & \\
    & x_1 \geq 0, \, x_2 \leq 0, \, x_3 \quad \text{libre}. &
    \end{array}
    \quad \text{y} \quad
    \begin{array}{lll}
    \text{min} & 6y_1 + 4y_2 & \\
    \text{s.a.} & y_1 + 2y_2 \geq 3 & \\
    & 2y_1 + y_2 \leq 4 & \\
    & 3y_1 + 2y_2 = 8 & \\
    & y_1 \leq 0, \, y_2 \quad \text{libre}. &
    \end{array}
    \]

    \vspace{0.3cm}
    \textbf{Algunas observaciones adicionales:}
    \begin{itemize}
        \item Restricción primal “\(\leq\)” \(\Rightarrow\) Variable dual “\(\geq 0\)”.
        \item Restricción primal “\(\geq\)” \(\Rightarrow\) Variable dual “\(\leq 0\)”.
        \item Restricción primal “\(\textcolor{blue}{=}\)” \(\Rightarrow\) Variable dual libre.
    \end{itemize}
\end{frame}

\begin{frame}{La Regla General}
    \textbf{En general, si el PL primal es:}
    \[
    \begin{array}{lll}
    \text{max} & c_1x_1 + c_2x_2 + c_3x_3 & \\
    \text{s.a.} & A_{11}x_1 + A_{12}x_2 + A_{13}x_3 \quad \text{\textcolor{blue}{$\geq$}} \quad b_1 & \\
    & A_{21}x_1 + A_{22}x_2 + A_{23}x_3 \quad \text{\textcolor{blue}{$\leq$}} \quad b_2 & \\
    & A_{31}x_1 + A_{32}x_2 + A_{33}x_3 \quad \text{\textcolor{blue}{$=$}} \quad b_3 & \\
    & x_1 \geq 0, \, x_2 \leq 0, \, x_3 \quad \text{libre}. &
    \end{array}
    \]

    \vspace{0.3cm}
    \textbf{Su PL dual es:}
    \[
    \begin{array}{lll}
    \text{min} & b_1y_1 + b_2y_2 + b_3y_3 & \\
    \text{s.a.} & A_{11}y_1 + A_{21}y_2 + A_{31}y_3 \quad \text{\textcolor{blue}{$\geq$}} \quad c_1 & \\
    & A_{12}y_1 + A_{22}y_2 + A_{32}y_3 \quad \text{\textcolor{blue}{$\leq$}} \quad c_2 & \\
    & A_{13}y_1 + A_{23}y_2 + A_{33}y_3 \quad \text{\textcolor{blue}{$=$}} \quad c_3 & \\
    & y_1 \leq 0, \, y_2 \geq 0, \, y_3 \quad \text{libre}. &
    \end{array}
    \]

    \vspace{0.3cm}
    \textbf{Nota que la matriz de coeficientes de las restricciones está \textcolor{blue}{“traspuesta”}.}
\end{frame}

\begin{frame}{Representación Matricial}
    \textbf{En general, si el PL primal es:}
    \[
    \begin{array}{lll}
    \text{max} & c_1x_1 + c_2x_2 + c_3x_3 & \\
    \text{s.a.} & A_{11}x_1 + A_{12}x_2 + A_{13}x_3 \quad = \quad b_1 & \\
    & A_{21}x_1 + A_{22}x_2 + A_{23}x_3 \quad = \quad b_2 & \\
    & A_{31}x_1 + A_{32}x_2 + A_{33}x_3 \quad = \quad b_3 & \\
    & x_1 \geq 0, \, x_2 \geq 0, \, x_3 \geq 0, &
    \end{array}
    \]

    \vspace{0.3cm}
    \textbf{Está en la \textcolor{blue}{forma estándar}, su PL dual es:}
    \[
    \begin{array}{lll}
    \text{min} & b_1y_1 + b_2y_2 + b_3y_3 & \\
    \text{s.a.} & A_{11}y_1 + A_{21}y_2 + A_{31}y_3 \quad \geq \quad c_1 & \\
    & A_{12}y_1 + A_{22}y_2 + A_{32}y_3 \quad \geq \quad c_2 & \\
    & A_{13}y_1 + A_{23}y_2 + A_{33}y_3 \quad \geq \quad c_3. &
    \end{array}
    \]

    \vspace{0.3cm}
    \textbf{En la representación matricial:}
    \[
    \begin{array}{lll}
    \text{max} & c^T x & \\
    \text{s.a.} & Ax = b & \\
    & x \geq 0 &
    \end{array}
    \quad \text{y} \quad
    \begin{array}{lll}
    \text{min} & y^T b & \\
    \text{s.a.} & y^T A \geq c^T. &
    \end{array}
    \]
\end{frame}

\begin{frame}{El PL Dual para un PL Primal de Minimización}
    \textbf{Para un PL de minimización, su PL dual es \textcolor{blue}{maximizar el límite inferior}.}
    \textbf{Las reglas para las direcciones de las variables y restricciones están \textcolor{blue}{invertidas}:}
    
    \vspace{0.3cm}
    \[
    \begin{array}{lll}
    \text{min} & 3x_1 + \textcolor{blue}{4x_2} + 8x_3 & \\
    \text{s.a.} & x_1 + 2x_2 + 3x_3 \quad \text{\textcolor{blue}{$\geq$}} \quad 6 & \\
    & 2x_1 + x_2 + 2x_3 \quad \text{\textcolor{blue}{$\leq$}} \quad 4 & \\
    & x_1 \geq 0, \, x_2 \leq 0, \, x_3 \quad \text{libre}. &
    \end{array}
    \quad \Rightarrow \quad
    \begin{array}{lll}
    \text{max} & 6y_1 + 4y_2 & \\
    \text{s.a.} & y_1 + 2y_2 \leq 3 & \\
    & 2y_1 + y_2 \geq 4 & \\
    & 3y_1 + 2y_2 = 8 & \\
    & y_1 \geq 0, \, y_2 \leq 0. &
    \end{array}
    \]

    \vspace{0.3cm}
    \textbf{Nota que:}
    \[
    3x_1 + 4x_2 + 8x_3 \\
    \geq (y_1 + 2y_2)x_1 + (2y_1 + y_2)x_2 + (3y_1 + 2y_2)x_3 \\
    \geq (x_1 + 2x_2 + 3x_3)y_1 + (2x_1 + x_2 + 2x_3)y_2 \\
    \geq 6y_1 + 4y_2.
    \]
\end{frame}

\begin{frame}{La Regla General, Unicidad y Simetría}
    \textbf{La regla general para encontrar el PL dual:}

    \vspace{0.3cm}
    \[
    \begin{array}{|c|c|c|c|c|}
    \hline
    \textbf{Función Obj.} & \text{max} & \text{min} & \textbf{Función Obj.} \\
    \hline
    & \leq & \geq & \\
    \textbf{Restricción}& \geq & \leq & \textbf{Variable}\\
    & = & \text{urs.} & \\
    \hline
    & \geq 0 & \geq 0 & \\
    \textbf{Variable}& \leq 0 & \leq 0 & \textbf{Restricción}\\
    & \text{urs.} & = & \\
    \hline
    \end{array}
    \]

    \vspace{0.5cm}
    \textbf{Si el PL primal es un problema de maximización, hazlo de izquierda a derecha.}

    \textbf{Si el PL primal es un problema de minimización, hazlo de derecha a izquierda.}

    \vspace{0.5cm}
    \begin{block}{\textbf{Proposición 1 (Unicidad y Simetría de la Dualidad)}}
        \textit{Para cualquier PL primal, existe un único dual, cuyo dual es el primal.}
    \end{block}
\end{frame}

\begin{frame}{Ejemplos de Pares Primal-Dual}
    \textbf{Ejemplo 1:}

    \[
    \begin{array}{lll}
    \text{min} & 2x_1 + 3x_2 & \\
    \text{s.a.} & 4x_1 + x_2 \quad \leq \quad 9 & \\
    & x_1 \quad \geq \quad 6 & \\
    & 2x_1 - x_2 \quad \geq \quad 8 & \\
    & x_1 \leq 0, \, x_2 \quad \text{libre}. &
    \end{array}
    \quad \Leftrightarrow \quad
    \begin{array}{lll}
    \text{max} & 9y_1 + 6y_2 + 8y_3 & \\
    \text{s.a.} & 4y_1 + y_2 + 2y_3 \quad \geq \quad 2 & \\
    & y_1 - y_3 \quad = \quad 3 & \\
    & y_1 \leq 0, \, y_2 \geq 0, \, y_3 \geq 0. &
    \end{array}
    \]

    \vspace{0.5cm}
    \textbf{Ejemplo 2:}

    \[
    \begin{array}{lll}
    \text{max} & 3x_1 - x_2 & \\
    \text{s.a.} & x_1 + 2x_2 \quad = \quad 6 & \\
    & 3x_1 + 3x_2 \quad \leq \quad -4 & \\
    & x_1 \quad \text{libre}, \, x_2 \geq 0. &
    \end{array}
    \quad \Leftrightarrow \quad
    \begin{array}{lll}
    \text{min} & 6y_1 - 4y_2 & \\
    \text{s.a.} & y_1 + 3y_2 \quad = \quad 3 & \\
    & 2y_1 + 3y_2 \quad \geq \quad -1 & \\
    & y_1 \quad \text{libre}, \, y_2 \geq 0. &
    \end{array}
    \]
\end{frame}

\begin{frame}{Teoremas de Dualidad}
    \textbf{La dualidad proporciona muchas propiedades interesantes.}

    \vspace{0.3cm}
    \textbf{Ilustraremos estas propiedades para problemas lineales (PL) primales en \textbf{forma estándar}:}
    
    \vspace{0.5cm}
    \[
    \begin{array}{c}
    \text{max} \quad c^T x \\
    \text{s.a.} \quad Ax = b \\
    \quad x \geq 0
    \end{array}
    \quad \Leftrightarrow \quad
    \begin{array}{c}
    \text{min} \quad y^T b \\
    \text{s.a.} \quad y^T A \geq c^T.
    \end{array}
    \quad (1)
    \]

    \vspace{0.5cm}
    \textbf{Se puede demostrar que todas las propiedades que introduciremos se aplican a otros pares primal-dual.}
\end{frame}

\begin{frame}{Dualidad Débil}
    \[
    \begin{array}{c}
    \text{max} \quad c^T x \\
    \text{s.a.} \quad Ax = b \\
    \quad x \geq 0
    \end{array}
    \quad \Leftrightarrow \quad
    \begin{array}{c}
    \text{min} \quad y^T b \\
    \text{s.a.} \quad y^T A \geq c^T.
    \end{array}
    \]

    \vspace{0.3cm}
    \textbf{El PL dual proporciona un \textcolor{blue}{límite superior} para el PL primal.}

    \vspace{0.5cm}
    \begin{block}{\textbf{Proposición 2 (Dualidad Débil)}}
        \textit{Para los PL definidos en (1), si \( x \) e \( y \) son factibles para el primal y el dual, entonces \( c^T x \leq y^T b \).}
    \end{block}

    \vspace{0.3cm}
    \textbf{Demostración.} Mientras \( x \) e \( y \) sean factibles para el primal y el dual, tenemos:

    \[
    c^T x \leq y^T A x \quad (x \geq 0 \, \text{y} \, y^T A \geq c^T) \\
    \leq y^T b \quad (Ax = b).
    \]

    \vspace{0.3cm}
    \textbf{Por lo tanto, se cumple la dualidad débil.}
\end{frame}

\begin{frame}{Suficiencia de la Optimalidad}
    \[
    \begin{array}{c}
    \text{max} \quad c^T x \\
    \text{s.a.} \quad Ax = b \\
    \quad x \geq 0
    \end{array}
    \quad \Leftrightarrow \quad
    \begin{array}{c}
    \text{min} \quad y^T b \\
    \text{s.a.} \quad y^T A \geq c^T.
    \end{array}
    \]

    \vspace{0.3cm}
    \textbf{Ahora tenemos una \textcolor{blue}{condición suficiente} para soluciones óptimas.}

    \vspace{0.5cm}
    \begin{block}{\textbf{Proposición 3 (Condición Suficiente para la Optimalidad)}}
        \textit{Si \(\bar{x}\) y \(\bar{y}\) son factibles para el primal y el dual y \( c^T \bar{x} = \bar{y}^T b \), entonces \(\bar{x}\) y \(\bar{y}\) son óptimos para el primal y el dual.}
    \end{block}

    \vspace{0.3cm}
    \textbf{Demostración.} Para cualquier \( y \) factible para el dual, tenemos \( c^T \bar{x} \leq y^T b \) por dualidad débil. Pero nos dan que \( c^T \bar{x} = \bar{y}^T b \), así que tenemos \( \bar{y}^T b \leq y^T b \) para cualquier \( y \) factible para el dual. Esto solo nos dice que \(\bar{y}\) es óptimo para el dual. Para \(\bar{x}\), es lo mismo.

    \vspace{0.3cm}
    \textbf{Dada una solución factible para el primal \(\bar{x}\), si podemos encontrar una solución factible para el dual tal que sus valores objetivos sean \textbf{idénticos}, entonces \(\bar{x}\) es óptima.}
\end{frame}

\begin{frame}{La Solución Óptima Dual}
    \[
    \begin{array}{c}
    \text{max} \quad c^T x \\
    \text{s.a.} \quad Ax = b \\
    \quad x \geq 0
    \end{array}
    \quad \Leftrightarrow \quad
    \begin{array}{c}
    \text{min} \quad y^T b \\
    \text{s.a.} \quad y^T A \geq c^T.
    \end{array}
    \]

    \vspace{0.3cm}
    \textbf{Si hemos resuelto el PL primal, la \textcolor{blue}{solución óptima dual} está allí.}

    \vspace{0.5cm}
    \begin{block}{\textbf{Proposición 4 (Solución Óptima Dual)}}
        \textit{Para los PL definidos en (1), si \(\bar{x}\) es óptima primal con la base \( B \), entonces \(\bar{y}^T = c_B^T A_B^{-1}\) es óptima dual.}
    \end{block}

    \vspace{0.3cm}
    \textbf{Demostración.} Dado que \( B \) es óptima, los costos reducidos \( c_B^T A_B^{-1} A_N - c_N^T \geq 0 \). Como \( c_B^T = c_B^T A_B^{-1} A_B \), tenemos

    \[
    \bar{y}^T A = c_B^T A_B^{-1} A = c_B^T A_B^{-1} \left[ A_B \quad A_N \right] \geq \left[ c_B^T \quad c_N^T \right] = c^T.
    \]

    \vspace{0.3cm}
    Por lo tanto, \(\bar{y}\) es factible para el dual. Como \(\bar{y}^T b = c_B^T A_B^{-1} b = c_B^T x_B = c^T x\), \(\bar{x}\) y \(\bar{y}\) tienen el mismo valor objetivo y son ambas óptimas.
\end{frame}

\begin{frame}{Dualidad Fuerte}
    \[
    \begin{array}{c}
    \text{max} \quad c^T x \\
    \text{s.a.} \quad Ax = b \\
    \quad x \geq 0
    \end{array}
    \quad \Leftrightarrow \quad
    \begin{array}{c}
    \text{min} \quad y^T b \\
    \text{s.a.} \quad y^T A \geq c^T.
    \end{array}
    \]

    \vspace{0.3cm}
    \textbf{El hecho de que \( c_B^T A_B^{-1} \) sea óptimo dual implica \textcolor{blue}{dualidad fuerte}.}

    \vspace{0.5cm}
    \begin{block}{\textbf{Proposición 5 (Dualidad Fuerte)}}
        \textit{Para los PL definidos en (1), \(\bar{x}\) y \(\bar{y}\) son óptimos primal y dual si y solo si \(\bar{x}\) y \(\bar{y}\) son factibles para el primal y el dual y \( c^T \bar{x} = \bar{y}^T b \).}
    \end{block}

    \vspace{0.3cm}
    \textbf{Demostración.} Para probar esta afirmación de si y solo si:

    \begin{itemize}
        \item \((\Leftarrow)\): Por la Proposición 3.
        \item \((\Rightarrow)\): Dado que \( c_B^T A_B^{-1} \) es una solución óptima dual, el valor objetivo óptimo dual es \( c_B^T A_B^{-1} b \), que es igual al valor objetivo óptimo primal \( c^T \bar{x} \).
    \end{itemize}

    \vspace{0.3cm}
    Como \(\bar{y}\) es óptimo dual, \(\bar{y}^T b = c_B^T A_B^{-1} b = c^T \bar{x}\).

    \vspace{0.5cm}
    \textit{Nota:} Como el PL dual puede o no tener una solución óptima única, \(\bar{y}^T\) y \( c_B^T A_B^{-1} \) pueden o no ser idénticos. En cualquier caso, la afirmación se cumple.
\end{frame}

\begin{frame}{Implicaciones de la Dualidad Fuerte}
    \textbf{La dualidad fuerte ciertamente implica dualidad débil.}
    
    \begin{itemize}
        \item La dualidad débil dice que el PL dual proporciona un límite.
        \item La dualidad fuerte dice que el límite es \textcolor{blue}{ajustado}, es decir, no se puede mejorar.
    \end{itemize}
    
    \vspace{0.3cm}
    \textbf{Los PL primales y duales son \textcolor{blue}{equivalentes}.}

    \vspace{0.3cm}
    \textbf{Dado el resultado de un PL, podemos predecir el resultado de su dual:}

    \[
    \begin{array}{|c|c|c|c|}
    \hline
    \textbf{Primal} & \textbf{Dual} & & \\
    & \text{Inviable} & \text{No Acotado} & \text{Óptimo Finito} \\
    \hline
    \text{Inviable} & \checkmark & \checkmark & \times \\
    \text{No Acotado} & \checkmark & \times & \times \\
    \text{Óptimo Finito} & \times & \times & \checkmark \\
    \hline
    \end{array}
    \]

    \vspace{0.5cm}
    \begin{itemize}
        \item \(\checkmark\) significa posible, \(\times\) significa imposible.
        \item Primal no acotado \(\Rightarrow\) sin límite superior \(\Rightarrow\) dual inviable.
        \item Primal óptimo finito \(\Rightarrow\) valor objetivo finito \(\Rightarrow\) dual óptimo finito.
        \item Si el primal es inviable, el dual aún puede ser inviable (según ejemplos).
    \end{itemize}
\end{frame}

\begin{frame}{Ejemplo}
    \textbf{Consideremos los siguientes PL primal y dual:}

    \[
    \begin{array}{lll}
    \text{max} & x_1 & \\
    \text{s.a.} & 2x_1 - x_2 \leq 4 & \\
    & 2x_1 + x_2 \leq 8 & \quad \Leftrightarrow \quad \text{min} \quad 4y_1 + 8y_2 + 3y_3 \\
    & x_2 \leq 3 & \text{s.a.} & 2y_1 + 2y_2 \geq 1 \\
    & x_j \geq 0 \quad \forall j = 1, 2. & & -y_1 + y_2 + y_3 \geq 0 \\
    & & & y_i \geq 0 \quad \forall i = 1, \ldots, 3.
    \end{array}
    \]

    \vspace{0.3cm}
    \textbf{Para el PL primal en \textbf{forma estándar}, tenemos:}

    \[
    c^T = \left[ 1 \quad 0 \quad 0 \quad 0 \quad 0 \right] \quad \text{y} \quad A = \left[ \begin{array}{ccccc}
    2 & -1 & 1 & 0 & 0 \\
    2 & 1 & 0 & 1 & 0 \\
    0 & 1 & 0 & 0 & 1
    \end{array} \right]
    \]

    \vspace{0.5cm}
    \textbf{Resolvamos el PL primal para obtener una \textcolor{blue}{solución óptima dual}.}
\end{frame}

\begin{frame}{Solución Óptima Primal}
    \textbf{Utilizando el método símplex, obtenemos el siguiente \textbf{tablao óptimo}:}

    \[
    \left[ \begin{array}{cccccc|c}
    -1 & 0 & 0 & 0 & 0 & 0 & 0 \\
    2 & -1 & 1 & 0 & 0 & 0 & x_3 = 4 \\
    2 & 1 & 0 & 1 & 0 & 0 & x_4 = 8 \\
    0 & 1 & 0 & 0 & 0 & 1 & x_5 = 3
    \end{array} \right]
    \quad \rightarrow \ldots \rightarrow \quad
    \left[ \begin{array}{cccccc|c}
    0 & 0 & \frac{1}{4} & \frac{1}{4} & 0 & 0 & 3 \\
    1 & 0 & \frac{1}{4} & \frac{1}{4} & 0 & 0 & x_1 = 3 \\
    0 & 1 & -\frac{1}{2} & \frac{1}{2} & 0 & 0 & x_2 = 2 \\
    0 & 0 & \frac{1}{2} & -\frac{1}{2} & 1 & 0 & x_5 = 1
    \end{array} \right]
    \]

    \vspace{0.5cm}
    \textbf{La base óptima asociada es \( B = (1, 2, 5) \).}

    \vspace{0.3cm}
    \textbf{La solución óptima primal es \(\bar{x} = (3, 2)\).}

    \vspace{0.3cm}
    \textbf{El valor objetivo asociado es \( z^* = 3 \).}
\end{frame}

\begin{frame}{Solución Óptima Dual}
    \textbf{Recordemos que:}

    \[
    c^T = \left[ 1 \quad 0 \quad 0 \quad 0 \quad 0 \right] \quad \text{y} \quad A = \left[ \begin{array}{ccccc}
    2 & -1 & 1 & 0 & 0 \\
    2 & 1 & 0 & 1 & 0 \\
    0 & 1 & 0 & 0 & 1
    \end{array} \right].
    \]

    \vspace{0.5cm}
    \textbf{Dado \( x_B = (x_1, x_2, x_5) \) y \( x_N = (x_3, x_4) \), tenemos:}

    \[
    c_B^T = \left[ 1 \quad 0 \quad 0 \right] \quad \text{y} \quad A_B = \left[ \begin{array}{ccc}
    2 & -1 & 0 \\
    2 & 1 & 0 \\
    0 & 1 & 1
    \end{array} \right].
    \]
\end{frame}

\begin{frame}{Solución Óptima Dual}
    \textbf{Dada la base óptima primal, obtenemos una \textcolor{blue}{solución dual}:}

    \[
    \bar{y}^T = c_B^T A_B^{-1} = \left[ 1 \quad 0 \quad 0 \right] \left[ \begin{array}{ccc}
    \frac{1}{4} & \frac{1}{4} & 0 \\
    -\frac{1}{2} & \frac{1}{2} & 0 \\
    \frac{1}{2} & -\frac{1}{2} & 1
    \end{array} \right] = \left[ \frac{1}{4} \quad \frac{1}{4} \quad 0 \right].
    \]

    \vspace{0.5cm}
    \textbf{Para \(\bar{y} = \left( \frac{1}{4}, \frac{1}{4}, 0 \right)\):}

    \begin{itemize}
        \item Es factible para el dual: \( 2 \left( \frac{1}{4} \right) + 2 \left( \frac{1}{4} \right) \geq 1 \) y \( -\frac{1}{4} + \frac{1}{4} + 0 \geq 0 \).
        \item Su valor objetivo dual es \( w = 4 \left( \frac{1}{4} \right) + 8 \left( \frac{1}{4} \right) = 3 = z^* \).
    \end{itemize}

    \vspace{0.5cm}
    \textbf{Por lo tanto, \(\bar{y}\) es \textcolor{blue}{óptimo dual}.}
\end{frame}

\begin{frame}{Holgura Complementaria}
    \textbf{Consideremos \( v \), las \textcolor{blue}{variables de holgura} del PL dual:}

    \[
    \text{min} \quad y^T b \\
    \text{s.a.} \quad y^T A - v^T = c^T \quad (2) \\
    \quad v \geq 0.
    \]

    \vspace{0.5cm}
    \begin{block}{\textbf{Proposición 6 (Holgura Complementaria)}}
        \textit{Para los PL definidos en (1) y (2), \(\bar{x}\) y \((\bar{y}, \bar{v})\) son óptimos primal y dual si y solo si son factibles y \(\bar{v}^T \bar{x} = 0\).}
    \end{block}

    \vspace{0.5cm}
    \textbf{Demostración.} Tenemos que \( c^T \bar{x} = \left( \bar{y}^T A - \bar{v}^T \right) \bar{x} = \bar{y}^T A \bar{x} - \bar{v}^T \bar{x} = \bar{y}^T b - \bar{v}^T \bar{x} \).

    \vspace{0.3cm}
    Por lo tanto, \(\bar{v}^T \bar{x} = 0\) si y solo si \( c^T \bar{x} = \bar{y}^T b\), es decir, \(\bar{x}\) y \((\bar{y}, \bar{v})\) son óptimos primal y dual según la dualidad fuerte.

    \begin{itemize}
        \item Nota que \(\bar{v}^T \bar{x} = 0\) si y solo si \(\bar{v}_i \bar{x}_i = 0\) para todo \( i \) ya que \(\bar{x} \geq 0\) y \(\bar{v} \geq 0\).
        \item Si una restricción dual (respectivamente primal) no es activa, la variable primal (respectivamente dual) correspondiente es \textbf{cero}.
    \end{itemize}
\end{frame}

\begin{frame}{Ejemplo}
    \textbf{Consideremos los PL primal y dual que hemos mencionado:}

    \[
    \begin{array}{lll}
    \text{max} & x_1 & \\
    \text{s.a.} & 2x_1 - x_2 \leq 4 & \\
    & 2x_1 + x_2 \leq 8 & \quad \Leftrightarrow \quad \text{min} \quad 4y_1 + 8y_2 + 3y_3 \\
    & x_2 \leq 3 & \text{s.a.} & 2y_1 + 2y_2 \geq 1 \\
    & x_j \geq 0 \quad \forall j = 1, 2. & & -y_1 + y_2 + y_3 \geq 0 \\
    & & & y_i \geq 0 \quad \forall i = 1, \ldots, 3.
    \end{array}
    \]
\end{frame}

\begin{frame}{Ejemplo}
    \textbf{Sean \( s_i \) y \( v_j \) las variables de holgura para los PL primal y dual:}

    \[
    \begin{array}{lll}
    \text{max} & x_1 & \\
    \text{s.a.} & 2x_1 - x_2 + s_1 = 4 & \\
    & 2x_1 + x_2 + s_2 = 8 & \\
    & x_2 + s_3 = 3 & \\
    & x_j \geq 0 \quad \forall j = 1, 2, \quad s_i \geq 0 \quad \forall i = 1, \ldots, 3.
    \end{array}
    \]

    \vspace{0.5cm}
    \[
    \text{min} \quad 4y_1 + 8y_2 + 3y_3
    \]

    \vspace{0.5cm}
    \[
    \begin{array}{lll}
    \text{s.a.} & 2y_1 + 2y_2 - v_1 = 1 & \\
    & -y_1 + y_2 + y_3 - v_2 = 0 & \\
    & y_i \geq 0 \quad \forall i = 1, \ldots, 3, \quad v_j \geq 0 \quad \forall j = 1, 2.
    \end{array}
    \]
\end{frame}

\begin{frame}{Ejemplo}
    \textbf{Sea \((\bar{x}, \bar{s})\) óptimo primal, tenemos \((\bar{x}, \bar{s}) = (3, 2, 0, 0, 1)\). Vamos a encontrar una solución óptima dual \((\bar{y}, \bar{v})\) sin resolver el PL dual.}

    \vspace{0.5cm}
    \begin{itemize}
        \item De acuerdo con la \textbf{holgura complementaria}, \(\bar{x}_1 > 0\), \(\bar{x}_2 > 0\) y \(\bar{s}_3 > 0\) implican que \(\bar{v}_1 = 0\), \(\bar{v}_2 = 0\) y \(\bar{y}_3 = 0\), respectivamente.
    \end{itemize}

    \vspace{0.5cm}
    \textbf{Las dos igualdades funcionales del dual se reducen a:}

    \[
    \begin{cases}
    2\bar{y}_1 + 2\bar{y}_2 = 1 \\
    -\bar{y}_1 + \bar{y}_2 = 0.
    \end{cases}
    \]

    \vspace{0.5cm}
    \textbf{Resolver estas ecuaciones resulta en \(\bar{y}_1 = \frac{1}{4}\) y \(\bar{y}_2 = \frac{1}{4}\). \((\bar{y}, \bar{v})\) es entonces óptimo dual garantizado.}

    \vspace{0.5cm}
    \textbf{Nota que \( z^* = 3 = w^* \).}
\end{frame}

\begin{frame}{¿Por qué dualidad?}
    \textbf{¿Por qué dualidad?} Dado un PL:
    \begin{itemize}
        \item Podemos resolverlo directamente.
        \item O podemos resolver el PL dual y luego obtener la solución óptima primal.
    \end{itemize}

    \vspace{0.5cm}
    \textbf{¿Por qué molestarse?}

    \vspace{0.3cm}
    El tiempo de cálculo del método símplex es aproximadamente proporcional a \( m^3 \).
    \begin{itemize}
        \item \( m \) es el número de restricciones funcionales del PL original.
        \item Y \( n \), el número de variables del PL original, no importa mucho.
    \end{itemize}

    \vspace{0.5cm}
    Si \( m \gg n \), resolver el PL dual puede tomar un tiempo significativamente \textcolor{blue}{más corto} que resolver el primal.

    \vspace{0.5cm}
    Hay muchos otros beneficios de tener dualidad. Veremos algunos más en este curso.
\end{frame}

\end{document}
