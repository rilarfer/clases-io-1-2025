\documentclass{article}
\usepackage{amsmath}
\usepackage{amssymb}
\usepackage{geometry}
\usepackage{xcolor}
\usepackage{mdframed}
\geometry{a4paper, margin=1in}

\title{Análisis Postóptimo}
\author{Ricardo Largaespada}
\date{07 de octubre 2025}

\newmdenv[
  backgroundcolor=blue!5,
  linecolor=blue,
  linewidth=1pt,
  roundcorner=5pt,
  skipabove=\baselineskip,
  skipbelow=\baselineskip
]{problem}

\begin{document}

\maketitle

\begin{problem}
Una empresa fabrica dos productos, \textbf{P1} y \textbf{P2}, utilizando dos recursos limitados: \textbf{mano de obra} y \textbf{material}. La siguiente tabla muestra los recursos requeridos por unidad de producto y la disponibilidad total de los recursos:

\begin{center}
\begin{tabular}{lccc}
\toprule
\textbf{Recurso} & \textbf{Disponibilidad} & \textbf{Requerimiento por P1} & \textbf{Requerimiento por P2} \\
\midrule
Mano de obra (horas) & 400 horas & 2 horas & 1 hora \\
Material (kg)        & 300 kg    & 1 kg    & 2 kg  \\
\bottomrule
\end{tabular}
\end{center}

Los beneficios por unidad son: \textbf{P1}: \$50 por unidad, \textbf{P2}: \$40 por unidad.

\textbf{a)} Formule y resuelva un modelo de programación lineal para determinar cuántas unidades de cada producto deben producirse para maximizar el beneficio total, considerando las restricciones de recursos.

\textbf{b)} La empresa está considerando introducir un nuevo producto, \textbf{P3}.

\begin{itemize}
    \item \textbf{Datos de P3:}
    \begin{itemize}
        \item \textbf{Beneficio por unidad:} \$45
        \item \textbf{Requerimientos de recursos por unidad:} Mano de obra: 1 hora, Material: 1 kg.
    \end{itemize}
\end{itemize}

\textbf{Tareas:}

\begin{enumerate}
    \item \textbf{Agregar la nueva variable \( x_3 \) correspondiente al producto P3 al modelo de programación lineal original.}

    \item \textbf{Encontrar la columna de restricciones asociada a la nueva variable \( x_3 \) (producto P3) en el modelo. Es decir, identificar los coeficientes de \( x_3 \) en cada restricción existente.}

    \item \textbf{Determinar si es rentable introducir el producto P3 al modelo, calculando el costo reducido de la nueva variable.}

    \item \textbf{Explicar si la nueva actividad (producción del producto P3) incrementará el beneficio total, aplicando la condición de optimalidad.}
\end{enumerate}

\textbf{c)} La empresa tiene una restricción adicional: debido a compromisos contractuales, debe producir al menos 100 unidades combinadas de \textbf{P1} y \textbf{P2}.\\

\textbf{Tareas:}

\begin{enumerate}
    \item \textbf{Formule el modelo de programación lineal incluyendo la nueva restricción.}
    \item \textbf{Resuelva el modelo y determine la cantidad óptima de cada producto a producir.}
    \item \textbf{Calcule el beneficio total óptimo.}
    \item \textbf{Analice cómo la nueva restricción afecta la solución óptima y discuta su impacto en el beneficio total.}
\end{enumerate}

\end{problem}
\end{document}
