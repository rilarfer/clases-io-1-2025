\documentclass[11pt]{article}

% --------- Idioma y tipografía ----------
\usepackage[spanish, es-nodecimaldot]{babel}
\usepackage[T1]{fontenc}
\usepackage[utf8]{inputenc}
\usepackage{microtype}
\usepackage{newpxtext,newpxmath} % Palatino + matemáticas
\usepackage[a4paper,margin=1in]{geometry}

% --------- Matemática y utilidades -------
\usepackage{amsmath,mathtools}

% --------- Colores, enlaces, refs --------
\usepackage{xcolor}
\definecolor{accent}{RGB}{10,80,160}
\usepackage[
  colorlinks,
  linkcolor=accent,
  urlcolor=accent,
  citecolor=accent
]{hyperref}
\usepackage[nameinlink]{cleveref}

% --------- Encabezados/Pies --------------
\usepackage{fancyhdr}
\pagestyle{fancy}
\fancyhf{}
\lhead{\textit{Problemas de Programación Lineal — Grupo B}}
\rhead{Ricardo Largaespada}
\cfoot{\thepage}
\setlength{\headheight}{14pt}

% --------- Tablas bonitas ----------------
\usepackage{booktabs,tabularx}
\renewcommand{\arraystretch}{1.15}

% --------- Cajas para "Problema" ---------
\usepackage[most]{tcolorbox}
\tcbset{
  boxsep=5pt,
  arc=2mm,
  colback=gray!2!white,
  colframe=accent!60!black,
  boxrule=0.6pt
}
\newtcbtheorem[auto counter]{problema}{Problema}{
  fonttitle=\bfseries,
  coltitle=white
}{prob}

% --------- Título ------------------------
\title{\textbf{Problemas de Programación Lineal — Grupo B}}
\author{Ricardo Largaespada}
\date{\today}

\begin{document}
\maketitle
\thispagestyle{fancy}

% ---------------------------------------------------
\begin{problema}{Método gráfico}{}
Resuelva mediante el método gráfico el siguiente modelo de \emph{maximización}:
\begin{alignat*}{2}
\text{Maximizar}\quad & Z = 11x_1 + 9x_2 \\
\text{sujeto a}\quad
& x_1 + 2x_2 &&\le 16,\\
& 2x_1 + x_2 &&\le 14,\\
-& x_1 + x_2 &&\le 5,\\
& x_1,\; x_2 &&\ge 0.
\end{alignat*}
\end{problema}

% ---------------------------------------------------
\begin{problema}{Cuotas de venta (Primo Seguros)}{}
La compañía de seguros \emph{Primo} planea introducir dos líneas: \emph{riesgo especial} ($x_1$) e \emph{hipotecas} ($x_2$).
Las ganancias unitarias son de \$7 por riesgo especial y \$3 por hipoteca.
Formule un modelo de programación lineal que \textbf{maximice} la ganancia total y resuélvalo por el método gráfico.

\begin{center}
\begin{tabularx}{\textwidth}{@{}l *{2}{>{\centering\arraybackslash}X} >{\centering\arraybackslash}X @{}}
\toprule
\textbf{Departamento} & \textbf{Riesgo especial (h/u)} & \textbf{Hipoteca (h/u)} & \textbf{Disponibles (h)}\\
\midrule
Suscripciones & 2 & 3 & 2000\\
Administración & 1 & 0 & 700\\
Reclamaciones & 0 & 1 & 600\\
\bottomrule
\end{tabularx}
\end{center}

\end{problema}

% ---------------------------------------------------
\begin{problema}{Desarrollo de aplicaciones móviles}{}
Una empresa desarrolla dos tipos de aplicaciones móviles, tipo 1 ($x_1$) y tipo 2 ($x_2$).
La aplicación tipo 1 requiere el doble de tiempo de desarrollo que la tipo 2. 
Si todo el tiempo disponible se dedica a la tipo 2, se pueden producir 420 aplicaciones tipo 2 al día.
Los límites de mercado para tipo 1 y tipo 2 son 160 y 180 por día, respectivamente.
Las ganancias son \$9 (tipo 1) y \$4 (tipo 2).
Formule el modelo y determine las cantidades de cada tipo que \textbf{maximizan} la ganancia total.
\end{problema}

% =================== VERIFICACIÓN DOCENTE ===================
\iffalse
VERIFICACIÓN (no imprimir al estudiante)

Problema 1:
Restricciones:
R1: x1 + 2x2 ≤ 16
R2: 2x1 + x2 ≤ 14
R3: -x1 + x2 ≤ 5
x1, x2 ≥ 0

Vértices factibles (intersecciones enteras):
(0,0), (7,0) [R2 con eje], (0,5) [R3 con eje], (4,6) [R1∩R2].
R2∩R3 = (?,?) no factible (viola R2), R1∩R3 = (2,7).

Evaluación FO Z=11x1+9x2:
(0,0)->0; (7,0)->77; (0,5)->45; (4,6)->98; (2,7)->85.
Óptimo único en (x1,x2)=(4,6) con Z*=98.

Problema 2:
Variables: x1=riegos especiales, x2=hipotecas.
FO: Max Z=7x1+3x2
Sujeto a:
2x1+3x2 ≤ 2000   (Suscripciones)
x1 ≤ 700         (Administración)
x2 ≤ 600         (Reclamaciones)
x1,x2 ≥ 0

Vértices factibles (enteros):
(0,0), (700,0), (0,600),
Intersecciones con 2x1+3x2=2000:
- con x1=700 → x2=200 → (700,200) factible
- con x2=600 → x1=100 → (100,600) factible

Valores:
(700,200): Z=7*700+3*200=4900+600=5500
(100,600): Z=700+1800=2500
(700,0): 4900; (0,600):1800; (0,0):0
Óptimo único en (700,200) con Z*=5500.

Problema 3:
Tiempo total equivalente: si todo es tipo 2 → 420 ⇒
Restricción de tiempo: 2x1 + x2 ≤ 420.
Mercado: x1 ≤ 160, x2 ≤ 180.
FO: Max Z=9x1+4x2; x1,x2 ≥ 0.

Vértices relevantes:
(0,0), (160,0), (0,180),
Intersecciones con 2x1+x2=420:
- con x1=160 → x2=100 ⇒ (160,100) factible
- con x2=180 → x1=120 ⇒ (120,180) factible
(160,180) es infeasible (2*160+180=500>420).

Valores:
(160,100): Z=9*160+4*100=1440+400=1840
(120,180): Z=9*120+4*180=1080+720=1800
(160,0):1440; (0,180):720; (0,0):0
Óptimo único en (160,100) con Z*=1840.
\fi
% =================== FIN VERIFICACIÓN ===================

\end{document}
