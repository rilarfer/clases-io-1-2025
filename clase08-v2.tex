\documentclass[11pt]{article}

% --------- Idioma y tipografía ----------
\usepackage[spanish, es-nodecimaldot]{babel}
\usepackage[T1]{fontenc}
\usepackage[utf8]{inputenc}
\usepackage{microtype}
\usepackage{newpxtext,newpxmath} % Palatino + matemáticas
\usepackage[a4paper,margin=1in]{geometry}
\usepackage{caption}

% --------- Matemática y utilidades -------
\usepackage{amsmath,mathtools}

% --------- Colores, enlaces, refs --------
\usepackage{xcolor}
\definecolor{accent}{RGB}{10,80,160}
\usepackage[
  colorlinks,
  linkcolor=accent,
  urlcolor=accent,
  citecolor=accent
]{hyperref}
\usepackage[nameinlink]{cleveref}

% --------- Encabezados/Pies --------------
\usepackage{fancyhdr}
\pagestyle{fancy}
\fancyhf{}
\lhead{\textit{Problemas de Programación Lineal}}
\rhead{Ricardo Largaespada}
\cfoot{\thepage}
\setlength{\headheight}{14pt}

% --------- Tablas bonitas ----------------
\usepackage{booktabs,tabularx}
\renewcommand{\arraystretch}{1.12}

% --------- Cajas para "Problema" ---------
\usepackage[most]{tcolorbox}
\tcbset{
  boxsep=5pt,
  arc=2mm,
  colback=gray!2!white,
  colframe=accent!60!black,
  boxrule=0.6pt
}
\newtcbtheorem[auto counter]{problema}{Problema}{
  fonttitle=\bfseries,
  coltitle=white
}{prob}

% --------- Entorno de solución (toggle) ----------
\newif\ifshowsolutions
\showsolutionstrue % pon \showsolutionsfalse para ocultarlas

\newenvironment{solucion}[1][]%
{\ifshowsolutions\begin{tcolorbox}[title=Solución, breakable,
  colback=accent!6!white, colframe=accent!80!black,
  boxrule=0.6pt, fonttitle=\bfseries, #1]}%
{\end{tcolorbox}\fi}

% --------- Título ------------------------
\title{\textbf{Problemas de Programación Lineal}}
\author{Ricardo Largaespada}
\date{\today}

\begin{document}
\maketitle
\thispagestyle{fancy}

% =====================================================
\begin{problema}{Proyecto multi-recurso (3 variables)}{}
Un ingeniero de sistemas debe decidir cuántos \emph{lotes de tareas} desarrollar de tres proyectos (\textbf{P1}, \textbf{P2}, \textbf{P3}). Cada lote requiere horas de \textbf{Programación}, \textbf{Pruebas} e \textbf{Implementación}. El beneficio unitario por lote es \$60 para P1, \$50 para P2 y \$55 para P3. Las disponibilidades de horas son 18 (Programación), 8 (Pruebas) y 14 (Implementación).

\[
\begin{array}{|c|c|c|c|c|}
\hline
 & \textbf{P1} & \textbf{P2} & \textbf{P3} & \textbf{Disponibilidad} \\
\hline
\textbf{Programación (h)} & 1 & 3 & 2 & 18 \\
\textbf{Pruebas (h)} & 1 & 1 & 2 & 8 \\
\textbf{Implementación (h)} & 2 & 1 & 1 & 14 \\
\hline
\textbf{Beneficio unitario (\$)} & 60 & 50 & 55 & \\
\hline
\end{array}
\]

\textbf{Tarea:} Formular el modelo de PL que \emph{maximice} el beneficio total y resolverlo con el método Simplex. (Variables $\ge 0$.)
\end{problema}

\begin{solucion}
\textbf{Modelo: } $\max Z=60x_1+50x_2+55x_3$ s.a.
\(
\begin{aligned}
x_1+3x_2+2x_3+s_1&=18,\\
x_1+x_2+2x_3+s_2&=8,\\
2x_1+x_2+x_3+s_3&=14,\qquad x_1,x_2,x_3,s_1,s_2,s_3\ge 0.
\end{aligned}
\)

\small
\begin{table}
\caption*{P1 — Iteración 1}
\begin{tabular}{l|rrrrrr|r}
\toprule
\textbf{Variable básica} & $X_1$ & $X_2$ & $X_3$ & $s_1$ & $s_2$ & $s_3$ & \textbf{RHS}\\
\midrule
\textbf{Fila 0 (FO)} & 60 & 50 & 55 & 0 & 0 & 0 & 0\\
$s_1$ & 1 & 3 & 2 & 1 & 0 & 0 & 18\\
$s_2$ & 1 & 1 & 2 & 0 & 1 & 0 & 8\\
$s_3$ & 2 & 1 & 1 & 0 & 0 & 1 & 14\\
\bottomrule
\end{tabular}
\end{table}

\begin{table}\centering
\caption*{P1 — Iteración 2 (entra $X_1$, sale $s_3$)}
\begin{tabular}{l|rrrrrr|r}
\toprule
\textbf{Variable básica} & $X_1$ & $X_2$ & $X_3$ & $s_1$ & $s_2$ & $s_3$ & \textbf{RHS}\\
\midrule
\textbf{Fila 0 (FO)} & 60 & 50 & 55 & 0 & 0 & 0 & 0\\
$s_1$ & 0 & 2.5 & 1.5 & 1 & 0 & -0.5 & 11\\
$s_2$ & 0 & 0.5 & 1.5 & 0 & 1 & -0.5 & 1\\
$X_1$ & 1 & 0.5 & 0.5 & 0 & 0 & 0.5 & 7\\
\bottomrule
\end{tabular}
\end{table}

\begin{table}\centering
\caption*{P1 — Iteración 3 (entra $X_3$, sale $s_2$)}
\begin{tabular}{l|rrrrrr|r}
\toprule
\textbf{Variable básica} & $X_1$ & $X_2$ & $X_3$ & $s_1$ & $s_2$ & $s_3$ & \textbf{RHS}\\
\midrule
\textbf{Fila 0 (FO)} & 60 & 50 & 55 & 0 & 0 & 0 & 0\\
$s_1$ & 0 & 2 & 0 & 1 & -1 & 0 & 10\\
$X_3$ & 0 & 0.3333 & 1 & 0 & 0.6667 & -0.3333 & 0.6667\\
$X_1$ & 1 & 0.3333 & 0 & 0 & -0.3333 & 0.6667 & 6.6667\\
\bottomrule
\end{tabular}
\end{table}

\begin{table}\centering
\caption*{P1 — Iteración 4 (óptimo; entra $X_2$, sale $X_3$)}
\begin{tabular}{l|rrrrrr|r}
\toprule
\textbf{Variable básica} & $X_1$ & $X_2$ & $X_3$ & $s_1$ & $s_2$ & $s_3$ & \textbf{RHS}\\
\midrule
\textbf{Fila 0 (FO)} & 60 & 50 & 55 & 0 & 0 & 0 & 0\\
$s_1$ & 0 & 0 & -6 & 1 & -5 & 2 & 6\\
$X_2$ & 0 & 1 & 3 & 0 & 2 & -1 & 2\\
$X_1$ & 1 & 0 & -1 & 0 & -1 & 1 & 6\\
\bottomrule
\end{tabular}
\end{table}

\noindent\textbf{Solución óptima:} $\boxed{x_1=6,\ x_2=2,\ x_3=0}$, $\boxed{Z^\star=460}$. \textbf{Holguras:} $s_1=6,\ s_2=0,\ s_3=0$.
\end{solucion}

% =====================================================
\begin{problema}{Asignación de horas con presupuestos (3 variables)}{}
Un ingeniero tiene 100 horas disponibles para distribuir entre tres proyectos: \textbf{A}, \textbf{B} y \textbf{C}. Cada hora dedicada a los proyectos incurre en costos de \textbf{recursos} y \textbf{mano de obra} como sigue: A (recursos \$4, mano de obra \$20), B (recursos \$6, mano de obra \$10), C (recursos \$3, mano de obra \$15). Los presupuestos máximos son \$480 para recursos y \$1500 para mano de obra. El ingreso por hora de trabajo es \$110 (A), \$150 (B) y \$120 (C).

\[
\begin{array}{|c|c|c|c|c|}
\hline
 & \textbf{Proyecto A} & \textbf{Proyecto B} & \textbf{Proyecto C} & \textbf{Disponibilidad / Límite} \\
\hline
\textbf{Tiempo total (h)} & 1 & 1 & 1 & 100 \\
\textbf{Recursos (USD/h)} & 4 & 6 & 3 & 480 \\
\textbf{Mano de Obra (USD/h)} & 20 & 10 & 15 & 1500 \\
\hline
\textbf{Ingreso (USD/h)} & 110 & 150 & 120 & \\
\hline
\end{array}
\]

\textbf{Tarea:} Determinar cuántas horas asignar a A, B y C para \emph{maximizar} el ingreso total, respetando los límites de tiempo y presupuesto. (Variables $\ge 0$.)
\end{problema}


% =====================================================
\begin{problema}{Producción de aplicaciones (3 variables)}{}
Una empresa produce tres aplicaciones: \textbf{App A}, \textbf{App B} y \textbf{App C}. Las utilidades por unidad son \$7 (A), \$10 (B) y \$9 (C). Cada unidad requiere horas en tres etapas: \textbf{Desarrollo}, \textbf{Pruebas} y \textbf{Diseño UX}. Las disponibilidades son 200 horas de Desarrollo, 240 de Pruebas y 120 de Diseño UX.

\[
\begin{array}{|c|c|c|c|c|}
\hline
 & \textbf{App A} & \textbf{App B} & \textbf{App C} & \textbf{Disponibilidad} \\
\hline
\textbf{Desarrollo (h/unidad)} & 4 & 5 & 3 & 200 \\
\textbf{Pruebas (h/unidad)} & 6 & 3 & 4 & 240 \\
\textbf{Diseño UX (h/unidad)} & 2 & 4 & 3 & 120 \\
\hline
\textbf{Utilidad (USD/unidad)} & 7 & 10 & 9 & \\
\hline
\end{array}
\]

\textbf{Tarea:} Formular el modelo que \emph{maximice} la utilidad total y resolverlo con Simplex. (Variables $\ge 0$.)
\end{problema}


\end{document}
