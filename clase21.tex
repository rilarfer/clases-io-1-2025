\documentclass{beamer}
\usepackage[utf8]{inputenc}
\usepackage{amsmath}

\title{Problema de Transporte: Representación Gráfica y Formulación Matemática}
\author{Ricardo Largaespada}
\date{\today}

\begin{document}

\frame{\titlepage}

\begin{frame}{Objetivo de la Clase}
    \begin{itemize}
        \item Comprender y describir la estructura de un problema de transporte.
        \item Formular el problema de transporte como un modelo matemático.
        \item Aplicar la representación gráfica para resolver problemas básicos de transporte.
    \end{itemize}
\end{frame}

\begin{frame}{Descripción General del Problema de Transporte}
    \begin{itemize}
        \item \textbf{Ejemplo:} Distribuir productos desde varios almacenes a diferentes tiendas.
        \item \textbf{Objetivo:} Minimizar el costo de transporte.
        \item \textbf{Elementos clave:} Orígenes (almacenes), destinos (tiendas), rutas y costos.
    \end{itemize}
\end{frame}

\begin{frame}{Representación Gráfica}
    \begin{itemize}
        \item \textbf{Nodos:} Representan los puntos de origen y destino.
        \item \textbf{Arcos:} Conectan los orígenes y destinos, mostrando rutas y costos.
        \item \textbf{Ejemplo:} Graficar un esquema de transporte simple.
    \end{itemize}
    \begin{center}
        \includegraphics[width=0.8\textwidth]{ruta_de_transporte.png} % Imagen ilustrativa
    \end{center}
\end{frame}

\begin{frame}{Modelo Matemático del Problema de Transporte}
    \textbf{Variables:}
    \begin{itemize}
        \item $x_{ij}$: Cantidad transportada desde el origen $i$ al destino $j$.
    \end{itemize}
    \textbf{Función Objetivo:}
    \begin{equation*}
        \min \sum_{i} \sum_{j} c_{ij} x_{ij}
    \end{equation*}
    \textbf{Restricciones:}
    \begin{itemize}
        \item Restricción de oferta: $\sum_{j} x_{ij} \leq s_i$
        \item Restricción de demanda: $\sum_{i} x_{ij} \geq d_j$
        \item $x_{ij} \geq 0$
    \end{itemize}
\end{frame}

\begin{frame}{Pregunta Orientadora}
    \textit{"¿Cómo podríamos representar un problema de distribución de productos entre varios destinos de manera que podamos encontrar la solución más eficiente?"}
\end{frame}

\begin{frame}{Actividad Independiente: Formulación y Representación Gráfica}
    \textbf{Descripción:} Resuelve un problema de transporte básico que incluye la formulación matemática y la representación gráfica.
    \begin{itemize}
        \item Incluir varios orígenes y destinos.
        \item Representar gráficamente el problema y formular matemáticamente.
    \end{itemize}
\end{frame}

\begin{frame}{Recursos y Medios}
    \begin{itemize}
        \item \textbf{Medios:} Pizarra, Marcadores, Entorno Virtual de Aprendizaje (EVA).
        \item \textbf{Recursos:} Diapositivas, Hojas de trabajo, Ejercicios prácticos.
    \end{itemize}
\end{frame}

\begin{frame}{Conclusión}
    \begin{itemize}
        \item El problema de transporte permite optimizar la distribución de productos.
        \item La representación gráfica facilita la comprensión del modelo matemático.
        \item Aplicar estos conceptos ayuda a resolver problemas prácticos en logística y distribución.
    \end{itemize}
\end{frame}

\end{document}