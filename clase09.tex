\documentclass{beamer}

\usepackage[utf8]{inputenc}
\usepackage[spanish]{babel}
\usepackage{amsmath}
\usepackage[nosetup]{evan}
%\usetheme{Goddard}
\usetheme{Madrid}
\hypersetup{colorlinks,allcolors=.,urlcolor=magenta}
\usepackage[table]{xcolor} % Para definir colores en tablas
\usepackage{graphicx} % Para redimensionar la tabla
\usepackage{multicol}
\title{Variantes del Método Simplex}
\subtitle{Uso de Variables Artificiales: Método de la ``M''}
\author[Ricardo Largaespada]{Ricardo Jesús Largaespada Fernández}
\institute[UNI]{Ingeniería de Sistemas, DACTIC, UNI}
\date{\today}

\begin{document}

\frame{\titlepage}

\section{Objetivos de Aprendizaje}

\begin{frame}{Objetivos de Aprendizaje}
    \begin{itemize}
        \item Comprender el concepto de variables artificiales y su uso en el Método Simplex para resolver problemas de programación lineal.
        \item Aplicar el Método de la ``M'' para resolver problemas de programación lineal con restricciones de igualdad.
    \end{itemize}
\end{frame}

\section{Inicio}

\begin{frame}{Inicio}
    \textbf{Actividades:}
    \begin{itemize}
        \item Tomar asistencia.
        \item Pregunta orientadora: \textit{¿Qué sabemos sobre variables artificiales y cómo se utilizan en el Método Simplex?}
    \end{itemize}
\end{frame}

\section{Desarrollo}

\begin{frame}{Variables Artificiales en el Método Simplex}
    \begin{itemize}
        \item Las \textbf{variables artificiales} se introducen para resolver problemas con restricciones de igualdad o mayores que ($= \text{ o }\ge$).
        \item Permiten transformar el problema en una forma adecuada para el Método Simplex.
    \end{itemize}
    \pause
    \textbf{Ejemplo:} \\
    \textit{Problema con restricciones de igualdad.}
    \begin{align*}
        \text{Minimizar} & \quad Z = 2x_1 + 3x_2 \\
        \text{Sujeto a} & \quad x_1 + x_2 = 5 \\
        & \quad x_1, x_2 \geq 0
    \end{align*}
    \pause
    Se introduce la variable artificial \( A_1 \) para manejar la restricción de igualdad.
\end{frame}

\begin{frame}{Método de la ``M''}
    \begin{itemize}
        \item El \textbf{Método de la ``M''} se utiliza cuando se introducen variables artificiales.
        \item Se agrega una gran penalización (\( M \)) en la función objetivo para forzar la eliminación de las variables artificiales en la solución óptima.
    \end{itemize}
    \pause
    \textbf{Ejemplo:}
    \begin{align*}
        \text{Minimizar} & \quad Z = 2x_1 + 3x_2 + MA_1 \\
        \text{Sujeto a} & \quad x_1 + x_2 + A_1 = 5 \\
        & \quad x_1, x_2, A_1 \geq 0
    \end{align*}
\end{frame}

\begin{frame}{Método de la M Grande}
    \textbf{Paso 1}: Expresa el problema en la forma estándar.
    \newline
    \textbf{Paso 2}: Agrega una variable artificial no negativa al lado izquierdo de cada una de las ecuaciones correspondientes a restricciones del tipo '$\ge$' o '='.
    \newline
    Esta dificultad se elimina introduciendo una condición que asegura que las variables sean cero en la solución final (si existe una solución).
    \newline
    \textbf{Paso 3}: Por último, utiliza las variables artificiales como la solución inicial y continúa con la rutina simplex habitual hasta que se obtenga la solución óptima.
\end{frame}

\begin{frame}{Ejemplo I}
    Encuentra la solución óptima para el siguiente PPL:
    \[
    \text{Maximizar } Z = 5x_1 + 12x_2 + 4x_3
    \]
    \[
    \text{sujeto a:}
    \]
    \[
    x_1 + 2x_2 + x_3 \leq 5
    \]
    \[
    2x_1 + x_2 + 3x_3 = 2
    \]
    \[
    x_1 \geq 0, \, x_2 \geq 0, \, x_3 \geq 0
    \]
\end{frame}

\begin{frame}
Conversión a la forma estándar
    \[
    Z = 5x_1 + 12x_2 + 4x_3 - M A_1
    \]
    \[
    x_1 + 2x_2 + x_3 + S_1 = 5
    \]
    \[
    2x_1 + x_2 + 3x_3 + A_1 = 2
    \]
    \[
    A_1 = 2 - 2x_1 - x_2 - 3x_3
    \]
    Sustituye en Z:
    \[
    Z = 5x_1 + 12x_2 + 4x_3 - M(2 - 2x_1 - x_2 - 3x_3)
    \]
    Sustitución en Z y Tablas Simplex:
    \(
    Z = (5 + 2M)x_1 + (12 + M)x_2 + (4 + 3M)x_3 - 2M
    \)
    \newline
    \(
    x_1 + 2x_2 + x_3 + S_1 = 5
    \)
    \newline
    \(
    2x_1 + x_2 + 3x_3 + A_1 = 2
    \)
\end{frame}

\begin{frame}{Tabla Simplex Inicial}
    \[
    \begin{array}{|c|c|c|c|c|c|c|}
    \hline
    \text{Var. Básica} & x_1 & x_2 & x_3 & S_1 & A_1 & \text{SOLUCIÓN} \\
    \hline
    Z & -5 - 2M & -12 - M & -4 - 3M & 0 & 0 & -2M \\
    S_1 & 1 & 2 & 1 & 1 & 0 & 5 \\
    A_1 & 2 & 1 & 3 & 0 & 1 & 2 \\
    \hline
    \end{array}
    \]

    Ingresa la variable $x_3$ y sale de la base la variable $A_1$. El elemento pivote es 3.
\end{frame}

\begin{frame}{Iteración 1}
    \[
    \begin{array}{|c|c|c|c|c|c|c|}
    \hline
    Z & -7/3 & -32/3 & 0 & 0 & 4/3 + M & 8/3 \\
    S_1 & 1/3 & 5/3 & 0 & 1 & -1/3 & 13/3 \\
    x_3 & 2/3 & 1/3 & 1 & 0 & 1/3 & 2/3 \\
    \hline
    \end{array}
    \]

    Ingresa la variable $x_2$ y sale de la base la variable $x_3$. El elemento pivote es 1/3.
\end{frame}

\begin{frame}{Iteración Final}
    \[
    \begin{array}{|c|c|c|c|c|c|c|}
    \hline
    Z & 19 & 0 & 32 & 0 & 12 + M & 24 \\
    S_1 & -3 & 0 & -5 & 1 & -2 & 1 \\
    x_2 & 2 & 1 & 3 & 0 & 1 & 2 \\
    \hline
    \end{array}
    \]

    La solución óptima es $z = 24$\\
    $x_1= 0, x_2= 2, x_3= 0, S_1= 1, A_1= 0$.
\end{frame}

\begin{frame}{Ejemplo II}
    Encuentra la solución óptima para el siguiente PPL:
    \[
    \text{Minimizar } z = 4x_1 + x_2
    \]
    \[
    \text{Sujeto a:}
    \]
    \[
    3x_1 + x_2 = 3
    \]
    \[
    4x_1 + 3x_2 \geq 6
    \]
    \[
    x_1 + 2x_2 \leq 4
    \]
    \[
    x_1 \geq 0, \, x_2 \geq 0
    \]
\end{frame}

\begin{frame}
    Convierte a la forma estándar:
    \[
    \text{Minimizar } z = 4x_1 + x_2 + MA_1 + MA_2
    \]
    \[
    \text{Sujeto a:}
    \]
    \[
    3x_1 + x_2 + A_1 = 3
    \]
    \[
    4x_1 + 3x_2 - S_1 + A_2 = 6
    \]
    \[
    x_1 + 2x_2 + S_2 = 4
    \]
    \[
    x_1, x_2, S_1, S_2, A_1, A_2 \geq 0
    \]
\end{frame}

\begin{frame}
    Despejar el valor de variables artificiales:
    \begin{align*}
        A_1 & = 3 -3x_1 -x_2\\
        A_2 & = 6 -4x_1 -3x_2 + S_1
    \end{align*}
    Sustituye en \(Z\):
    \begin{align*}
    Z &= 4x_1 + x_2 + M(3 - 3x_1 - x_2) + M(6 - 4x_1 - 3x_2 + S_1)\\
    &= (4 - 7M)x_1 + (1 - 4M)x_2 + MS_1 + 9M\\
    &= (4 - 7M)x_1 + (1 - 4M)x_2 - MS_1 + 9M
    \end{align*}
\end{frame}

\begin{frame}{Tabla Simplex}
\begin{table}
\centering
\resizebox{10cm}{!} {
    $\begin{tabular}{|c|c|c|c|c|c|c|c|}
    \hline
    \text{Var. Básica} & X_1 & X_2 & S_1 & S_2 & R_1 & R_2 & \text{Solución} \\
    \hline
    Z & -4 + 7M & -1 + 4M & -M & 0 & 0 & 0 & 9M \\
    R_1 & 3 & 1 & 0 & 0 & 1 & 0 & 3 \\
    R_2 & 4 & 3 & -1 & 0 & 0 & 1 & 6 \\
    S_2 & 1 & 2 & 0 & 1 & 0 & 0 & 4 \\
    \hline
    Z & 0 & 1/3 + 5/3M & -M & 0 & 4/3 - 7/3M & 0 & 4 + 2M \\
    X_1 & 1 & 1/3 & 0 & 0 & 1/3 & 0 & 1 \\
    R_2 & 0 & 5/3 & -1 & 0 & -4/3 & 1 & 2 \\
    S_2 & 0 & 5/3 & 0 & 1 & -1/3 & 0 & 3 \\
    \hline
    Z & 0 & 0 & 1/5 & 0 & 8/5 - M & -1/5 - M & 18/5 \\
    X_1 & 1 & 0 & 1/5 & 0 & 3/5 & -1/5 & 3/5 \\
    X_2 & 0 & 1 & -3/5 & 0 & -4/5 & 3/5 & 6/5 \\
    S_2 & 0 & 0 & 1 & 1 & 1 & -1 & 1 \\
    \hline
    Z & 0 & 0 & 0 & -1/5 & 7/5 - M & -M & 17/5 \\
    X_1 & 1 & 0 & 0 & -1/5 & 2/5 & 0 & 2/5 \\
    X_2 & 0 & 1 & 0 & 3/5 & -1/5 & 0 & 9/5 \\
    S_1 & 0 & 0 & 1 & 1 & 1 & -1 & 1 \\
    \hline
\end{tabular}$
}
\caption{Tabla Simplex Ejemplo II}
\end{table}
\end{frame}

\begin{frame}{Ejemplo III}
    Maximizar $Z = 3x_1 + 2x_2 + x_3$
    \newline
    Sujeto a:
    \newline
    $2x_1 + x_2 + x_3 = 12$
    \newline
    $3x_1 + 4x_3 = 11$
    \newline
    $x_2 \geq 0, \, x_3 \geq 0 \, \text{y} \, x_1 \, \text{no tiene restricciones.}$
\end{frame}

\begin{frame}{Tabla Simplex y Solución}
    \[
    x_1', x_1'', x_2, x_3, A_1, A_2 \geq 0
    \]
    \[
    \text{Maximizar } z = 3x_1'' - 3x_1' + 2x_2 + x_3 - M A_1 - M A_2
    \]
    Sujeto a:
    \begin{align*}
    2x_1'' - 2x_1' + x_2 + x_3 + A_1 &= 12\\
    3x_1'' - 3x_1' + 4x_3 + A_2 &= 11
    \end{align*}
\end{frame}

\begin{frame}{Tabla Simplex}
\begin{table}
\centering
$\resizebox{10cm}{!} {
    \begin{tabular}{|c|c|c|c|c|c|c|c|}
    \hline
    C_j & 3 & -3 & 2 & 1 & -M & -M & \\
    \hline
    \text{Var. Básica} & x_1' & x_1'' & x_2 & x_3 & A_1 & A_2 & \text{Solución} \\
    \hline
    A_1 & 2 & -2 & 1 & 1 & 1 & 0 & 12 \\
    A_2 & 3 & -3 & 0 & 4 & 0 & 1 & 11 \\
    \hline
    Z & -5M - 3 & 5M + 3 & -M - 2 & -5M - 1 & 0 & 0 & -23M \\
    \hline
    A_1 & 0 & 0 & 1 & -5/3 & 1 & -2/3 & 14/3 \\
    x_1' & 1 & -1 & 0 &  4/3 & 0 & 1/3 & 11/3 \\
    \hline
    Z & 0 & 0 & -M - 2 & 5/3M +3 & 0 & 5/3M +1 & -14/3M+11\\
    \hline
    x_2 & 0 & 0 & -5/3 & 1 & -2/3 & 14/3 \\
    x_1' & 1 & -1 & 0 & 4/3 & 0 & 1/3 & 11/3 \\
    \hline
    Z & 0 & 0 & 0 & -1/3 & M+2 & M-1/3 & 61/3 \\
    \hline
    x_2 & 5/4 & -5/4 & 1 & 0 & 1 & -1/4 & 37/4 \\
    x_3 & 3/4 & -3/4 & 0 & 1 & 0 & 1/4 & 11/4 \\
    \hline
    Z & 1/4 & -1/4 & 0 & 0 & M+2 & M-1/4 & 85/4 \\
    \hline
\end{tabular}
}$
\caption{Tabla Simplex Ejemplo III}
\end{table}

    El problema tiene solución ilimitada (no acotada). La variable $x_1''$ debe entrar a la base pero ninguna variable puede salir.
\end{frame}

\subsubsection{Problemas Propuestos}
\begin{frame}{Problemas Propuestos}
    \begin{itemize}
        \item \textbf{Problema 1:} Maximizamos \(Z = 3x_1 + 2x_2\), sujeto a:
        \begin{align*}
            2x_1 + x_2 &\leq 6 \\
            x_1 + 3x_2 &\geq 3 \\
            x_1, x_2 &\geq 0
        \end{align*}
        Resolver utilizando el Método de la "M".

        \item \textbf{Problema 2:} Minimizar \(Z = 2x_1 + 5x_2\), sujeto a:
        \begin{align*}
            x_1 + 2x_2 &= 4 \\
            3x_1 + 4x_2 &\geq 10 \\
            x_1, x_2 &\geq 0
        \end{align*}
        Resolver utilizando el Método Simplex con variables artificiales.
    \end{itemize}
\end{frame}

\section{Cierre}

\begin{frame}{Cierre}
    \begin{itemize}
        \item \textbf{Resumen:} Hemos aprendido a introducir variables artificiales y a aplicar el Método de la "M".
        \item \textbf{Trabajo independiente:} Resolver problemas adicionales utilizando el Método de la "M".
        \item \textbf{Próxima clase:} Resolveremos más problemas de programación lineal.
    \end{itemize}
\end{frame}

\end{document}
